\documentclass{beamer}

\input{../flat-blue-theme.inc}
\input{../footnotes.inc}

\usepackage[utf8]{inputenc}
\usepackage[OT1]{fontenc}
\usepackage[ngerman]{babel}
\usepackage{tikz}
\usepackage{lmodern}
\usepackage{tabularx}
\usepackage{hyperref}
\usepackage{xifthen}
\usepackage{multicol}
\usepackage{cleveref}
\usepackage{tikz}
\usetikzlibrary{positioning}
\usetikzlibrary{arrows}

\setbeamercovered{invisible}
\beamertemplatenavigationsymbolsempty
\renewcommand*\insertshorttitle
{
	% To not count \pause as new frame. And also to make the package "appendixnumberbeamer" work.
	\makebox[0.94\textwidth]{\oldmacro\hfill\insertframenumber\,/\,\inserttotalframenumber}
}
\condensedToc

\author[Hauke Stieler]{Hauke Stieler\\\href{mailto:4stieler@informatik.uni-hamburg.de}{4stieler@inf}}
\title{HowTo: Steuererklärung}
\date{13. November 2024}

\begin{document}
	{
		\setbeamertemplate{footline}{}
		\setbeamertemplate{headline}{}
		\setbeamertemplate{background}{\centering\includegraphics[width=\paperwidth,trim=0 0 0 0.5cm]{images/tax-government}}
		\addtocounter{page}{-1}
		\frame{}
	}

	{
		\setbeamertemplate{footline}{}
		\setbeamertemplate{headline}{}
		\maketitle
		\addtocounter{page}{-1}
	}
	
	\begin{frame}{Disclaimer}
		\begin{center}
			Diese Präsentation inklusive Vortrag ist keine Rechts-, Steuer- oder Finanzberatung!\n\pause
			Es besteht keinerlei Garantie für die Richtigkeit der Informationen in dieser Präsentation, alle Angaben ohne Gewähr!\n\pause
			Folien findet ihr im Wiki.
		\end{center}
	\end{frame}

	\begin{frame}
		\begin{center}
			\textbf{Was für Vorwissen hast du?}\\\pause
			\hfill\\
			Wer hat vor demnächst eine Steuererklärung zu machen?\\\pause
			\hfill\\
			Wer hat schon mal eine Steuererklärung gemacht?
		\end{center}
	\end{frame}

	\begin{frame}[t]{Inhalt}
		\tableofcontents[hidesubsections]
	\end{frame}
	
	\section{Basics}
	
		\begin{frame}[t]{Inhalt}
			\tableofcontents[currentsection,hideothersubsections]
		\end{frame}
	
		\subsection{Wieso? Weshalb? Warum?}
	
			\begin{frame}{Was sind Steuern?\citewiki{Steuer}}
				\begin{itemize}
					\item Zahlungen an Staat/Land/Gemeinde
					\item Kein Anspruch auf Gegenleistung
					\begin{itemize}
						\item Anders als bei Abgaben, Gebühren, Maut, etc.
						\item Beispiel: Fahrräder dürfen auf Kfz-Straßen fahren, obwohl sie keine Kfz-Steuer zahlen
					\end{itemize}
				\end{itemize}
			\end{frame}
		
			\begin{frame}{Warum eigentlich Steuern?\citewiki{Steuer\#Der\_Zweck\_der\_Steuererhebung}}
				\textbf{Staatshaushalt} decken.
				\begin{itemize}
					\item \href{https://www.bundeshaushalt.de/DE/Bundeshaushalt-digital/bundeshaushalt-digital.html}{bundeshaushalt.de}
					\item Straßen, Eisenbahn, ÖPNV, Zuschüsse zur Rente, Bildung, BAföG, Wettervorhersage, sämtliche Ämter/Verwaltungen
				\end{itemize}
			
				\textbf{Lenkung} von Verhalten (z.B. Tabaksteuer \textrightarrow\ Leute sollen weniger rauchen)
				
				\textbf{Umverteilung} von reich zu arm
			\end{frame}
			
			\begin{frame}
				\begin{center}
					\includegraphics[height=\textheight]{images/bundeshaushalt.png}
				\end{center}
			\end{frame}
		
			\begin{frame}{Von wem an wen werden Steuern gezahlt?}
				Steuerzahler zahlt Steuern an Bund/Land/Gemeinde:\n
				
				\begin{description}[An Gemeinde]
					\item[An Bund] Einkommensteuer, Lohnsteuer, Umsatzsteuer, ...
					\item[An Land] Erbschaftssteuer, Lotteriesteuer, Biersteuer, ...
					\item[An Gemeinde] Grundsteuer, Hundesteuer, ...
				\end{description}
				\n\pause
				Für später: Lohnsteuer ist Sonderform der Einkommenssteuer.
			\end{frame}
		
%		\subsection{Grundsätze}
%	
%			\begin{frame}{Maxime im Aufbau von Steuern\citewiki{Steuer\#Die\_Grundsätze\_der\_Besteuerung}}
%				\begin{description}[labelwidth=0cm]
%					\item[Gerechtigkeit] Nur wirtschaftliche Faktoren wichtig (nicht z.B. Hautfarbe)
%					\item[Gleichmäßigkeit] Kein Spielraum/Willkür
%					\item[Rückwirkungsverbot] Steuergesetze dürfen nicht rückwirkend in Kraft treten
%					\item[Ergiebigkeit] Steuern sollten Staatshaushalt decken + keinen zu hohen Verwaltungsaufwand erzeugen
%					\item[Unmerklichkeit] Steuererhebung und -belastung sollte man nicht merken
%					\item[Praktikabilität] Steuergesetze sollen transparent, bestimmt und einfach sein
%				\end{description}
%			\end{frame}
	
		\subsection{Steuersatz}
		
			\begin{frame}
				Steuer (in Prozent), die man zahlen muss. Kann sich wie folgt entwickeln\citewiki{Steuertarif}:\n
				\begin{description}
					\item[Proportional] Immer gleicher Prozentwert (z.B. 19\% Umsatzsteuer)
					\item[Progressiv] Prozentwert steigt mit Bemessungsgrundlage (z.B. Lohnsteuer)
					\item[Regressiv] Prozentwert sinkt mit Bemessungsgrundlage \vspace{0.1cm}\newline
					{\tiny Existiert in Deutschland nicht; In USA/UK sind Sozialabgaben regressiv\\}
					\item[Stufen] Prozentwert verändert sich Stufenweise
				\end{description}
			\end{frame}
			
			\begin{frame}{Steuerklassen}
				Pro Job hat man eine Steuerklasse:
				\begin{enumerate}[I]
					\item Ledig/geschieden
					\item Alleinerziehend
					\item Verheiratet, eine Person hat kein Einkommen (ist damit in Steuerklasse 5) \textrightarrow\ zusammengelegte Freibeträge
					\item Verheiratet, aber wie Steuerklasse 1
					\item Andere Person in Ehe hat Steuerklasse 3
					\item Pro weiteren Job, dort dann \textit{keine} Freibeträge!
				\end{enumerate}
			\end{frame}
	
		\subsection{Glossar}
	
			\begin{frame}
				\begin{description}[Steuerfestsetzung]
					\item[Steuerschuldner] Gesetzlich verpflichtet Steuern zu zahlen
					\item[Steuerträger] Wirtschaftlich belastet\citewiki{Direkte\_und\_indirekte\_Steuer}
					\item[Steuerzahler] Person, die tatsächlich das Geld überweist\citewiki{Steuerzahler}
					\pause
					\item[Veranlagung] Ermittlungsverfahren + Festsetzungsverfahren\citewiki{Veranlagung\_(Steuerrecht)}
					\item[Steuerfestsetzung] Verwaltung stellt Steuerbescheid aus\citewiki{Steuerfestsetzung}
					\pause
					\item[Steuerbescheid] Zettel auf dem steht, welche Steuern anfallen\citewiki{Steuerbescheid}
					\item[Bemessungsgrundlage] Wert auf dem Steuer basiert (z.B. zu versteuerndes Einkommen)\citewiki{Bemessungsgrundlage\_(Steuerrecht)}
				\end{description}
			\end{frame}
	
		\subsection{Steuerarten}
		
			\begin{frame}
				\begin{description}[Indirekte Steuer]
					\item[Direkte Steuer] Schuldner = Träger (z.B. Lohnsteuer)
					\item[Indirekte Steuer] Schuldner $\neq$ Träger (z.B. Umsatzsteuer\footnote{Umsatzsteuer = Mehrwertsteuer})
					\item[Pauschalsteuer] Fester Steuersatz (z.B. Umsatzsteuer)
					\item[Quellensteuer] Steuer wird direkt an der Quelle gezahlt (z.B. Lohnsteuer, Abgeltungssteuer)
				\end{description}\n
				Personensteuer, Realsteuer, Veranlagungssteuer, Individualsteuer, ...
			\end{frame}
			
			\begin{frame}
				\begin{center}
					\includegraphics[width=\textwidth]{images/steuersong.png}
				\end{center}
			\end{frame}
	
	\section{Einkommen \& Steuern}
	
		\begin{frame}[t]{Inhalt}
			\tableofcontents[currentsection,hideothersubsections]
		\end{frame}
	
		\subsection{Grundsätzliches}
			
			\begin{frame}{Nummern}
				\textbf{Steueridentifikationsnummer}
				\begin{itemize}
					\item Auch: \textit{Identifikationsnummer}, \textit{IdNr} oder \textit{Steuer-ID}
					\item Besteht aus 11 Zahlen
					\item Ein Leben lang gleich
					\item Zu finden: Einkommenssteuerbescheid, Lohnsteuerabrechnung, Schreiben vom Finanzamt
				\end{itemize}
				\pause
				\textbf{Steuernummer}
				\begin{itemize}
					\item Besteht aus 13 Zahlen
					\item Erstellt nach erster Einkommenssteuererklärung
					\item Ändert sich ggf. bei Umzug
					\item Zu finden: Einkommenssteuerbescheid
				\end{itemize}
			\end{frame}
		
			\begin{frame}{Art der Tätigkeit}
				\textbf{Selbständige Arbeit}
				\begin{itemize}
					\item Selbstständige
					\item Freiberufler (Freelancer)
					\item Unternehmer
				\end{itemize}
				\pause
				\textbf{Nichtselbständige Arbeit}
				\begin{itemize}
					\item Angestellte
					\item Werkstudis
				\end{itemize}
			\end{frame}
		
			\begin{frame}{Art der Tätigkeit}
				\begin{tabularx}{\linewidth}{X|X}
					\multicolumn{1}{c|}{\textbf{Selbständig}\citewiki{Selbständigkeit\_(beruflich)\#Steuerrecht}} &
					\multicolumn{1}{c}{\textbf{Angestellt}} \\[0.25cm]
					\begin{itemize}
						\item Schreibt Rechnungen
						\item Umsatzsteuer
						\item Steuererklärung \textbf{Pflicht}
						\item Frist bis 31.7.
					\end{itemize} &
					\begin{itemize}
						\item Festes Gehalt
						\item Lohnsteuer
						\item Steuererklärung \textbf{Optional}
						\item Vier Jahre Zeit
					\end{itemize}
				\end{tabularx}
			\end{frame}
		
		\subsection{Wie viel Steuern muss ich zahlen?}
			
			\begin{frame}{Wer bezahlt die Steuer?}
				Selbstständig
				\begin{itemize}
					\item Muss man selbst machen
					\item Ggf. Umsatzsteuervoranmeldung
				\end{itemize}
				\pause
				Nichtselbstständige
				\begin{itemize}
					\item Arbeitgeber überweist Steuer \textrightarrow\ Quellensteuer
					\item Du brauchst nichts weiter tun
				\end{itemize}
			\end{frame}
			
			\begin{frame}{Worauf muss ich Steuern zahlen?}
				Bruttolohn, oder?\pause\ Fast.
				\pause
				\begin{itemize}
					\item Jegliche Einnahmen (Lohn, Verkäufe\footnote{sog. privates Veräußerungsgeschäft}, Mieten, ...) und manche Ausgaben werden betrachtet
					\item Einnahmen $\neq$ Einkünfte $\neq$ Einkommen $\neq$ zu versteuerndes Einkommen\pause
					\item Grundfreibetrag (relevant für Gehalt)
					\begin{itemize}
						\item 2023: 10.908€ (909€ / Monat)
						\item 2024: 11.784€ (982€ / Monat)
					\end{itemize}
				\end{itemize}
			\end{frame}
			
			\begin{frame}{Zu ver-\\steuerndes\\Einkommen\\bestimmen}
				\begin{center}
					\begin{figure}
						\vspace{-2.75cm}
						\includegraphics[height=0.9\textheight]{images/zu-versteuerndes-einkommen}\citeurl{https://commons.wikimedia.org/wiki/File:Einnahmen_Einkuenfte_Einkommen_zvE.png}{Wikipedia (CC BY-SA 3.0)}
					\end{figure}
				\end{center}
			\end{frame}
			
			\begin{frame}{Wie viel Steuern muss ich zahlen?}
				Wie viel Steuern muss ich denn jetzt zahlen?\pause\ ... well\pause
				\begin{itemize}
					\item Steuersatz pro Euro berechnet\pause
					\item Grenzsteuersatz: Steuersatz auf den höchsten Euro Einkommen\pause
					\item Durchschnittssteuersatz: Steuersatz bezogen auf alle verdienten Euros\pause
					\item Steuersatz nach Zonen (je nach zu versteuerndem Einkommen)
				\end{itemize}
				\pause
				Und wie viel konkret?\pause\ Na so viel\citewiki[Einkommensteuer]{Einkommensteuer_(Deutschland)\#Tarif_2022}:\pause
				\begin{center}
					\includegraphics[width=\textwidth]{images/tarifzonen-formel}
				\end{center}
				\vspace*{-0.25cm}\pause
				{\tiny (Trivial!)}
				\vspace{0.55cm}
			\end{frame}
			
%			\begin{frame}
%			\begin{center}
%				\includegraphics[width=0.75\linewidth]{images/tarifzonen}
%			\end{center}
%			\end{frame}
		
			\begin{frame}{Wie viel Steuern muss ich zahlen?}
				\begin{center}
					\vspace{-0.5cm}
					\begin{figure}
						\includegraphics[width=0.9\linewidth]{images/tarifzonen-diagramm}\citeurl{https://commons.wikimedia.org/wiki/File:ESt_D_Grenz_und_Durchschnittsteuersatz_\%C2\%A732a_2022.svg}{Wikipedia (CC BY-SA 4.0)}
					\end{figure}
				\end{center}
			\end{frame}
			
			\begin{frame}{Wie viel Steuern muss ich zahlen?}
				\begin{center}
					\vspace{-0.5cm}
					\begin{figure}
						\includegraphics[width=0.9\linewidth]{images/tarifzonen-steuerfunktion-diagramm}\citeurl{https://commons.wikimedia.org/wiki/File:Linear_progressiver_Tarif_Steuerbetrag.svg}{Wikipedia (CC BY-SA 3.0)}
					\end{figure}
				\end{center}
			\end{frame}
			
			\begin{frame}{Exkurs: Steuern vs. Sozialabgaben}
				\begin{center}
					\vspace{-0.5cm}
					\begin{figure}
						\includegraphics[width=0.8\linewidth]{images/steuerrechner}\citeurl{https://www.finanzfluss.de/rechner/einkommensteuer/}{Finanzfluss -- Einkommensteuer berechnen}
					\end{figure}
				\end{center}
			\end{frame}
			
			\begin{frame}{Von der Steuer absetzen}
				\begin{center}
					\includegraphics[width=0.75\linewidth]{images/absetzen}
				\end{center}
				\pause
				Ausgaben mindern Steuerlast \textrightarrow\ Ausgaben entsprechend \textit{absetzen}
			\end{frame}
		
			\begin{frame}{Von der Steuer absetzen}
				\begin{itemize}
					\item Finanzamt weiß: Bruttolohn
					\item Finanzamt weiß nicht: Durch Arbeit entstandene Kosten
					\pause
					\item Beruflichs Ausgaben (sog. \textbf{Werbungskosten}):
					\begin{itemize}
						\item Fahrtkosten, Arbeitsmittel, ...
						\item Büromaterial, Internet, Telefon, Home-Office, ...
						\item Steuerberater, Steuersoftware, Lektüre, ...
						\item Haushaltsnahe Dienstleistungen, Spenden, ...
					\end{itemize}
					\item Ausgaben reduzieren \textit{zu versteuerndes Einkommen}
				\end{itemize}\n\pause
				
				\textbf{Nice:} Ausbildungskosten (auch fürs Studium) zählen als berufliche Kosten!
			\end{frame}
				
			\begin{frame}{Von der Steuer absetzen}
				\textbf{Problem:} Arbeitgeber hat Steuern ja schon gezahlt :(\n\pause
				\textbf{Lösung:} \textbf{Steuererklärung} \textrightarrow\ Dem Finanzamt \textit{nachträglich} über Ausgaben informieren und zu viel gezahlte Steuern zurück bekommen :)
			\end{frame}
		
		\subsection{Exkurs: Kapitalerträge}
		
			\begin{frame}
				\begin{itemize}
					\item Kapitalerträge = Zinsen, Dividenden, Kursgewinne, ...
					\item Kapitalerträge $\neq$ Einkommen \textrightarrow\ Kapitalertragssteuer\pause
					\item Kapitalertragssteuer = Quellensteuer = nice!
					\item Kapitalertragssteuer = 25\% + Soli + ggf. Kirchensteuer
					\begin{itemize}
						\item Grenzsteuersatz $<$ 25\%? \textrightarrow\ Günstigerprüfung
					\end{itemize}\pause
					\item Freibetrag (Sparerpauschbetrag)
					\begin{itemize}
						\item $\leq2022$: 801€
						\item $\geq2023$: 1000€
						\item Bei Bank/Broker "`Freistellungsauftrag"' festlegen
					\end{itemize}
				\end{itemize}\n
				\pause
				\textbf{Wo \& wie angeben?} \textrightarrow\ Anlage KAP
			\end{frame}
		
	\section{Studieren \& Steuern}
	
		\begin{frame}[t]{Inhalt}
			\tableofcontents[currentsection,hideothersubsections]
		\end{frame}
	
		\begin{frame}{Bachelor / Erstausbildung}
			\begin{itemize}
				\item Bachelor = Erstausbildung
				\begin{itemize}
					\item außer vorherige \textit{abgeschlossene} Ausbildung/Studium
				\end{itemize}
				\item Absetzen nur als Sonderausgaben (nicht als Werbungskosten!)
				\item Max. 6000€ pro Jahr
				\item Verlust/Ausgaben nicht übertragbar auf spätere Jahre :(
			\end{itemize}
		\end{frame}
	
		\begin{frame}{Master / Zweitausbildung}
			\begin{itemize}
				\item Master = Zweitausbildung
				\item Absetzen als Werbungskosten
				\item Unbegrenzt viel
				\item Verlust/Ausgaben übertragbar auf spätere Jahre
			\end{itemize}
		\end{frame}
	
		\begin{frame}{Was kann ich absetzen?}
			Kosten deines Studiums sind (anteilig) absetzbar.\n
			\begin{itemize}
				\item Semestergebühren, Kursgebühren, Exkursionen
				\item Bücher, Literatur, Leihgebühren, Software, ...
				\item Fahrtkosten + Unterkunft
				\item Arbeitsmittel, Druck-/Bindekosten, Hardware
				\item Telefon, Internet
				\item Spenden \& Mitgliedsbeiträge bei Vereinen
			\end{itemize}\n
			Siehe auch \href{https://www.elster.de/eportal/helpGlobal?themaGlobal=help_est_ufa_10_2023\#c9718}{ELSTER-FAQ}.
		\end{frame}
	
		\begin{frame}{BAföG und Steuern}
			\begin{itemize}
				\item BAföG = Zuschuss + Darlehen
				\item Beides steuerfrei \textrightarrow\ Kein Einfluss auf Steuern
				\item Nichts davon ist absetzbar
			\end{itemize}
		\end{frame}
	
	\section{Steuererklärung}
	
		\begin{frame}[t]{Inhalt}
			\tableofcontents[currentsection,hideothersubsections]
		\end{frame}
	
		\begin{frame}{Disclaimer 1}
			Im Folgenden \textbf{Fokus auf Angestellte \& Werkstudenten} (= nicht-selbstständige Arbeit \textrightarrow\ Lohnsteuer) und Leute nach/in \textbf{Zweitausbildung}.\n\pause
			\textbf{Freiberufler \& Selbstständige} \textrightarrow\ Umsatzsteuer (wird hier nicht betrachtet, man kann aber fast das gleiche absetzen, muss es nur woanders eintragen).\n\pause
			Ist man in der \textbf{Erstausbildung}, kann man das gleiche absetzen, aber in der Anlage zu Sonderausgaben.
		\end{frame}
		
		\begin{frame}{Disclaimer 2}
			Fokus auf das \textbf{Jahr 2023}, andere Jahre haben mindestens abweichende Beträge!
		\end{frame}
	
		\subsection{Aller Anfang ist schwer}

			\begin{frame}{Wie ist eine Steuererklärung aufgebaut}
				\begin{itemize}
					\item Hauptvordruck: Persönliche Daten, Bankverbindung, etc.
					\item Anlagen, z.B.
					\begin{itemize}
						\item N: \textbf{N}ichtselbstständige Arbeit (Angestellte \& Werkstudis)
						\item S: \textbf{S}elbstständige \& Freelancer
						\item Vorsorgeaufwand: Z.B. Beiträge zu Krankenkasse, Altersvorsorge, etc.
						\item Haushaltsnahe Aufwendungen: Wichtig für Wohnnebenkosten
						\item Sonderausgaben: Z.B. Spenden
					\end{itemize}
					\item Jede Anlage hat Zeilennummern. Beispiele:
					\begin{itemize}
						\item Hauptvordruck Zeile 10 = Vorname
						\item Anlage N Zeile 45 = Anzahl Tage für Homeoffice-Pauschale
					\end{itemize}
				\end{itemize}
			\end{frame}

			\begin{frame}{Wie/womit mache ich meine Steuererklärung?}
				\begin{itemize}
					\item ELSTER\footnote{Elster ... Selbstironie ist der Finanzbehörde nicht fremd.}: Online Dienst der Finanzverwaltung. Umsonst, schlichte Formulare, Dark-Mode.
					\item Steuersoftware: Günstig, IMHO selten Mehrwert gegenüber ELSTER
					\item Lohnsteuerhilfeverein: Nur Beratung, mittelmäßig teuer
					\item Steuerberater: Kann alles für dich machen, teuer
				\end{itemize}
			\end{frame}
		
			\begin{frame}{Welche Dokumente/Daten brauche ich?}
				\begin{itemize}
					\item Eigene Finanzen, Einnahmen/Ausgaben
					\item Relevante Rechnungen!
					\item Anwendbare Pauschalen kennen
					\item Fahrten zur Arbeit/Uni
					\item Tage mit mehr als 8h außerhalb
					\item Nebenkostenabrechnung
				\end{itemize}\n
				\textbf{Tipp:} Übers Jahr hinweg alles relevante sammeln und notieren.
			\end{frame}
				
			\begin{frame}{Welche Beträge gebe ich an?}
				\begin{itemize}
					\item Wenn ganze Euros gefordert: Immer zu eigenen Gunsten runden
					\begin{itemize}
						\item Bei Ausgaben: Aufrunden
						\item Bei Einnahmen: Abrunden
					\end{itemize}\pause
					\item Bei Einnahmen, Krankenkasse, SV, etc.: Tatsächliche Beträge
					\item Kosten, die nicht anteilig berücksichtig werden: Tatsächliche Beträge
					\item Kosten, die anteilig berücksichtigt werden: Anteilige Beträge
					\begin{itemize}
						\item Ausnahme: Haushaltsnahe Dienstleistungen
						\item Ich schriebe immer Prozentsatz in Kommentar
					\end{itemize}
				\end{itemize}
			\end{frame}
			
%			\begin{frame}{Wann mache ich meine Steuererklärung?}
%				\begin{itemize}
%					\item Immer im Folgejahr: Steuererklärung \textbf{für} 2022 macht man also 2023.
%				\end{itemize}\n
%				\textbf{Fristen:}\\
%				Angestellt? \textrightarrow\ freiwillige Abgabe \textrightarrow\ vier Jahre Zeit\\
%				Selbstständig/Freiberufler? \textrightarrow\ Pflicht \textrightarrow\ Bis 31. Juli\footnote{Coronabedingt ggf. länger}
%			\end{frame}
			
			\begin{frame}{Wie läuft das ab?}{}
				{\tiny (Timeline not to scale)}
				\begin{center}
					\hspace*{-0.8cm}
					\begin{tikzpicture}
					[
						>=stealth',
						label/.style={font={\scriptsize},align=center},
						dot/.style={circle,fill=black,minimum size=0.125cm,inner sep=0,outer sep=0.175cm},
					]
						\draw[->,line width=0.025cm] (0,0) -- (12,0);
%						\draw[line width=0.025cm] (0,-0.15) -- +(0,0.3);
						
						\visible<1->
						{
							\node[dot] (dot-1) at (1,0) {};
							\node[label,gray,below = 0cm of dot-1] {1.1.};
							\node[label,text width=2cm] (step-1) at (1,2) {Jahr beginnt, ab jetzt alle Ausgaben notieren};
							\draw[->] (step-1) -- (dot-1);
						}
						
						\visible<2->
						{
							\node[dot] (dot-2) at (4,0) {};
							\node[label,gray,below = 0cm of dot-2] {31.12.};
							\node[label,text width=2cm] (step-2) at (4,2) {Jahr endet, Lohnsteuerbescheid erhalten};
							\draw[->] (step-2) to[out=270,in=90] (dot-2);
							
							\draw[line width=0.05cm,red] (dot-1.center) -- (dot-2.center);
							\node[dot] at (dot-1) {};
							\node[dot] at (dot-2) {};
						}
						
						\visible<3->
						{
							\node[dot] (dot-3) at (5.5,0) {};
							\node[label,gray,below = 0cm  of dot-3] {Frühjahr};
							\node[label,text width=2cm] (step-3) at (5.5,3.5) {Steuererklärung vorbereiten};
							\draw[->] (step-3) -- (dot-3);
						}
						
						\visible<4->
						{
							\node[dot] (dot-4) at (7.5,0) {};
							\node[label,text width=2cm] (step-4) at (6.8,2) {Nebenkosten-abrechnung erhalten};
							\draw[->] (step-4) to[out=270,in=130] (dot-4);
						}
						
						\visible<5->
						{
							\node[dot] (dot-5) at (7.75,0) {};
							\node[label,text width=2cm] (step-5) at (7.9,3.5) {Steuererklärung vervollständigen \& abschicken};
							\draw[->] (step-5) to[out=270,in=80] (dot-5);
						}
						
						\visible<6->
						{
							\node[dot] (dot-6) at (9,0) {};
							\node[label,text width=2cm] (step-6) at (9,2) {Ggf. Rückfragen, Dokumente nachreichen};
							\draw[->] (step-6) to[out=270,in=90] (dot-6);
						}
						
						\visible<7->
						{
							\node[dot] (dot-7) at (11,0) {};
							\node[label,gray,below = 0cm of dot-7] {irgendwann ...};
							\node[label,text width=2cm] (step-7) at (11,2) {Einkommen-steuerbescheid erhalten};
							\draw[->] (step-7) -- (dot-7);
						}
					\end{tikzpicture}
				\end{center}
			\end{frame}
		
			\begin{frame}{Wie läuft das ab?}
				\begin{itemize}
					\item Jahr endet \& Erhalt Lohnsteuerbescheid
					\item Daten aus Lohnsteuerbescheinigung in Steuerprogramm eintragen / ggf. automatisch vorausgefüllt
					\item Ausgaben eintragen (s. nachfolgende Folien), ggf. auf Nebenkostenabrechnung warten
					\item Abschicken
					\item Ggf. Dokumente (z.B. Rechnungen) nachreichen
					\item Erhalt Einkommensteuerbescheid
					\begin{itemize}
						\item Prüfen \& nachvollziehen
						\item Enthält ggf. Erklärungen und Begründungen
					\end{itemize}
				\end{itemize}\n
				\textbf{Tipp:} So früh wie möglich (Januar/Februar) anfangen ;)
			\end{frame}
		
			\begin{frame}{Lohnsteuerbescheinigung}{}
				{\tiny Bekommt man vom Arbeitgeber}
				\begin{center}
					\includegraphics[width=0.65\linewidth]{images/lohnsteuerbescheinigung}
					\citeurl{https://www.bundesfinanzministerium.de/Content/DE/Downloads/BMF_Schreiben/Steuerarten/Lohnsteuer/2022-07-15-bekanntmachung-geaendertes-muster-ausdruck-elektronische-LSt-bescheinigung-2022.pdf}{bundesfinanzministerium.de}
				\end{center}
			\end{frame}
		
			\begin{frame}{Einkommensteuerbescheid}{}
				{\tiny Bekommt man vom Finanzamt nach abgegebener Steuererklärung}
				\begin{center}
					\includegraphics[width=0.55\linewidth]{images/einkommensteuerbescheid_2019}
					\citeurl{https://de.wikipedia.org/wiki/Datei:Einkommensteuerbescheid_2019.pdf}{commons.wikimedia.org}
				\end{center}
			\end{frame}
			
			\begin{frame}
				\begin{center}
					\includegraphics[height=0.9\textheight]{images/meme-tax-insane.jpg}
				\end{center}
			\end{frame}
		
		\subsection{Steuererklärung mit ELSTER}
		
			\begin{frame}{Los gehts}
				\begin{enumerate}
					\item Registrierung bei ELSTER
					\begin{itemize}
						\item Per Ausweis, Zertifikatsdatei, ...
					\end{itemize}\pause
					\item Login\pause
					\item Vorausgefüllte Steuererklärung
					\begin{itemize}
						\item Über Abruf v. Bescheinigungen, dauert ca. 1 Tag
					\end{itemize}
				\end{enumerate}
			\end{frame}
		
			\begin{frame}
				\begin{center}
					\vspace{-0.6cm}
					\hspace*{-0.91cm}
					\includegraphics[scale=0.24]{images/elster-1}
				\end{center}
			\end{frame}
		
			\begin{frame}
				\begin{center}
					\vspace{-0.6cm}
					\hspace*{-0.91cm}
					\includegraphics[scale=0.24]{images/elster-übersicht-1}
				\end{center}
			\end{frame}
		
			\begin{frame}
				\begin{center}
					\vspace{-0.6cm}
					\hspace*{-0.91cm}
					\includegraphics[scale=0.24]{images/elster-übersicht-2}
				\end{center}
			\end{frame}
		
		\subsection{Das kannst du absetzen}
		
			\begin{frame}{Pauschalen und Nachweise}
				Pauschalen:
				\begin{itemize}
					\item Werbungskosten
					\item Kontoführung: 16€
					\item Home-Office Pauschale
					\item Entfernungspauschale
					\item Verpflegungspauschale
				\end{itemize}
				Ohne Nachweise möglich\citeurl{https://www.finanzfluss.de/podcast/folge-165/}{Finanzfluss -- Podcast \#165}:
				\begin{itemize}
					\item Arbeitsmittel
					\item Telefon \& Internet
					\item Spenden
				\end{itemize}
			\end{frame}
			
			\begin{frame}{Werbungskostenpauschale}
				\begin{itemize}
					\item Je nach Jahr:
					\begin{itemize}
						\item $\leq2021$: 1000€
						\item $=2022$: 1200€
						\item $\geq2023$: 1230€
					\end{itemize}
					\item Arbeitgeber berücksichtigt das bereits!
					\begin{itemize}
						\item Braucht man nicht selbst eintragen
					\end{itemize}
					\item Weniger Werbungskosten als Pauschale?
					\begin{itemize}
						\item Wahrscheinlich keine Rückerstattung :(
						\item Trotzdem einreichen und wenn man Steuern nachzahlen muss zurückziehen (Fristen beachten!)
					\end{itemize}
				\end{itemize}
			\end{frame}
		
			\begin{frame}{Home-Office Pauschale}
				\begin{itemize}
					\item 6€ pro Arbeitstag im Home-Office
					\begin{itemize}
						\item $\leq2022$: Max. 120 Tage \& max. 600€
						\item $\geq2023$: Max. 210 Tage \& max. 1260€
					\end{itemize}
					\item Bereits in Werbungskostenpauschale enthalten!\pause
					\item Beispiele\footnote{Pauschbeträge für 2023}:
					\begin{itemize}
						\item Beispiel 1:\\
						700€ H.O. Pauschale + 200€ Werbungskosten = 900€\\
						Werbungskostenpauschale ist größer \textrightarrow\ Kein Vorteil\pause
						\item Beispiel 2:\\
						900€ H.O. Pauschale + 500€ Werbungskosten = 1400€\\
						1400€ - 1230€ = 170€ werden vom Finanzamt berücksichtigt
					\end{itemize}
				\end{itemize}
				\pause
				\textbf{Wo \& wie angeben?} \textrightarrow\ Anzahl an Tagen in Anlage N Zeile 61
			\end{frame}
		
			\begin{frame}{Fahrtkosten zur Arbeit/Uni}
				Stichwort: Entfernungspauschale / Pendlerpauschale\n
				
				\begin{itemize}
					\item Pro Kilometer 30ct (ab dem 21. Kilometer 35ct bzw. 38ct\footnote{Die 38ct ab dem 21. Kilometer gelten nur für die Jahre 2022 bis 2026})
					\item Einfache Wegstrecke (\textbf{nicht} hin + zurück)
					\begin{itemize}
						\item Nur bei Auswärtstätigkeit (s. Folie \ref{verpflegungsmehraufwand}) hin + zurück
					\end{itemize}
					\item Kilometer werden immer \textbf{ab}gerundet (z.B. 2,9km \textrightarrow\ 2km)
					\item Verkehrsmittel egal: Auto, Fahrrad, Fuß, Fahrgemeinschaft
					\item Zweck muss beruflich sein: Zum Job, zur Uni, zur Lerngruppe (auch bei jemandem Zuhause), zur OE, ...
				\end{itemize}\n\pause
				\textbf{Wo \& wie angeben?} \textrightarrow\ Anlage N Zeile 32 - 36\\\pause
				\textbf{Achtung:} Semesterticket wird woanders eingetragen!
			\end{frame}
			
			\begin{frame}{Fahrtkosten zur Arbeit/Uni}
				\begin{center}
					\includegraphics[height=0.85\textheight]{images/meme-fahrtkosten.jpg}
				\end{center}
			\end{frame}
		
			\begin{frame}{Semesterbeitrag (inkl. Semesterticket)}
				Als Werbungskosten, z.B. unter "`Fortbildungskosten"' oder "`Weitere Werbungskosten"'.\n
				\hfill\\\pause
				\textbf{Wo \& wie angeben?} \textrightarrow\ Anlage N Zeile 63 oder 64\n\pause
				\textbf{Hinweis:} Entfernungspauschale (Auto, Rad, Fuß) geht trotz Semesterticket trotzdem, da \textbf{unabhängig} vom Verkehrsmittel! Steuern sparen durch Rad fahren :D
			\end{frame}
		
			\begin{frame}{Steuertrick: Pendlerpauschale vs. Semesterticket}
				Annahme: Semesterticket kostet 150€ \textrightarrow\ 300€ pro Jahr.\n
				\textbf{Beispiel 1:}\\
				Mit ÖPNV 200km gependelt \textrightarrow\ 60€ Pendlerpauschale \textrightarrow\ Semesterticket angeben und die 60€ ignorieren\n
				\textbf{Beispiel 2:}\\
				Mit ÖPNV 1200km gependelt \textrightarrow\ 360€ Pendlerpauschale \textrightarrow\ Die 360€ angeben und Semesterbeiträg um je 150€ (also Preis des Semestertickets) kürzen
			\end{frame}
		
			\begin{frame}
				\begin{center}
					\vspace{-0.6cm}
					\hspace*{-0.91cm}
					\includegraphics[scale=0.24]{images/elster-fortbildungskosten-1}
				\end{center}
			\end{frame}
			
			\begin{frame}
				\begin{center}
					\vspace{-0.6cm}
					\hspace*{-0.91cm}
					\includegraphics[scale=0.24]{images/elster-fortbildungskosten-2}
				\end{center}
			\end{frame}
			
			\begin{frame}
				\begin{center}
					\vspace{-0.6cm}
					\hspace*{-0.91cm}
					\includegraphics[scale=0.24]{images/elster-fortbildungskosten-3}
				\end{center}
			\end{frame}
		
			\begin{frame}{Auswärtstätigkeit}
				Tätigkeit \textbf{nicht Zuhause \& nicht an erster Tätigkeitsstätte}.\n
				Erste Tätigkeitsstätte?
				\begin{itemize}
					\item Primärer Ort deiner Tätigkeit
					\item Pro Dienstverhältnis eine erste Tätigkeitsstätte
					\begin{itemize}
						\item Also ggf. zwei erste Tätigkeitsstätten (Uni \& Arbeitsplatz)
						\item Beispiel: Büro (Arbeitgeber) + Ikum (Uni)
					\end{itemize}
				\end{itemize}\n\pause
				Wirkt sich aus auf z.B.:
				\begin{itemize}
					\item Fahrtkostenpauschale (bei Auswährtstätigkeit Kilometer von Hin- + Rückweg angeben)
					\item Verpflegungsmehraufwand (s. nächste Folie)
				\end{itemize}
			\end{frame}
		
			\begin{frame}[label={verpflegungsmehraufwand}]{Verpflegungsmehraufwand}
				\begin{itemize}
					\item Auch: Verpflegungspauschale, Mehraufwand bei Verpflegung
					\item Essen kostet Geld \textrightarrow\ Pauschale für Verpflegung
					\item Nur bei Auswärtstätigkeit (Exkursion, Lerngruppe\footnote{Veranlassung primär beruflich, privater Grund für Treffen sollte absolut untergeordnet sein.}, OE, ...)
				\end{itemize}\n\pause
				Wie viel absetzen?
				\begin{itemize}
					\item Zeit ab Abfahrt von Zuhause bis Ankunft Zuhause
					\item $\leq$ 8h: Nix
					\item $>$ 8h und bei An-/Abreisetagen: 14€
					\item $>$ 24h: 28€ pro Tag {\tiny \textit{*hust* OEWE / NWE / EWE *hust*}}
				\end{itemize}\n\pause
				\textbf{Wo \& wie angeben?} \textrightarrow\ Anzahl Tage in Anlage N Zeile 75 - 77 
			\end{frame}
		
			\begin{frame}{Verpflegungsmehraufwand}
				\textbf{Beispiel 1:}\\
				7h am Hauptcamput zur Lerngruppe/Vorlesung/... = \textbf{ggf.} absetzbar (je nach dem wie lange man von Zuhause abwesend war)\n\pause
				\textbf{Beispiel 2:}\\
				8h am Ikum zur Lerngruppe/Vorlesung/... = \textbf{nicht} absetzbar\n\pause
				\textbf{Beispiel 3:}\\
				8h beim Kumpel zum Lernen = Absetzen möglich\n\pause
				\textbf{Beispiel 4:}\\
				OE = Offizielle Uni Veranstaltung \textrightarrow\ OEWE (Fr-So): 14€ für Anreisetag (Freitag) + 28€ (Samstag) + 14€ Abreisetag (Sonntag)
%				\textbf{Beispiel 3:}\\
%				8h in Bibliothek der UHH die außerhalb des Campus-Teils (z.B. Ikum) ist, in dem man normalerweise studiert und die somit räumlich vom Rest der ersten Tätigkeitsstätte getrennt ist \textrightarrow\ ... keine Ahnung, aber ein Versuch ists wert
			\end{frame}
		
			\begin{frame}{Arbeitsmittel}
				\begin{itemize}
					\item Berufsbekleidung, Equipment, Literatur für den Beruf, Hardware, Software
					\item Anschaffung, Reparatur, Miete, Reinigung
					\item Inoffizielle(!) Pauschale: 110€
					\item Mind. 10\% berufliche Nutzung, dann anteilig absetzbar
					\item Bis 952€\footnote{800€ Netto + 19\% Mehrwertsteuer}: Im entsprechenden Jahr absetzbar
					\item Über 952€: Verteilung über typische Nutzungsdauer
					\begin{itemize}
						\item Ab 2021 nicht mehr für Computer und Software
					\end{itemize}
				\end{itemize}\pause
				\textbf{Hinweis:} Unbedingt Rechnungen aufheben! Mindestens bis Erhalt des Steuerbescheids, besser ein paar Jahre länger.\\\pause
				\textbf{Wo \& wie angeben?} \textrightarrow\ Beträge + Artikelbeschreibungen in Anlage N Zeile 57
			\end{frame}
		
			\begin{frame}{Arbeitsmittel}
				\textbf{Beispiel 1:}\\
				Ich kaufe für 100€ ein Regal und 75\% der Bücher darin sind Fachliteratur \textrightarrow\ 75€ absetzbar.\n\pause
				\textbf{Beispiel 2:}\\
				Ich kaufe eine Tastatur, die ich nur beruflich nutze \textrightarrow\ komplette Kosten kann ich absetzen.\n\pause
				\textbf{Beispiel 3:}\\
				Ich habe nur 90€ für Arbeitsmittel ausgegeben :( \textrightarrow\ Einfach die 110€ Pauschale angeben :)
			\end{frame}
			
			\begin{frame}{Telefon \& Internet}
				\begin{itemize}
					\item Gilt für: Festnetz, Handy, Internet
					\item Pauschal 20\% der Kosten absetzbar (ohne Nachweis)
					\item Mehr als 20\% geht auch per Einzelnachweis:
					\begin{itemize}
						\item Repräsentativen 3-Monat-Zeitraum wählen
						\item Exakte Stunden erfassen (z.B. Zeiterfassung auf Arbeit)
						\item Private Nutzungsdauer ermitteln/schätzen
						\item Daraus beruflichen Anteil ermitteln
						\item Nachweise sehr lange aufbewahren (ca. 10 Jahre)!
					\end{itemize}
				\end{itemize}\n\pause
				\textbf{Wo \& wie angeben?} \textrightarrow\ Anteilige Kosten unter weitere Werbungskosten in Anlage N Zeile 64
			\end{frame}
		
			\begin{frame}{Nebenkosten}
				20\%\ von \textbf{haushaltsnahen Dienstleistungen}\citeurl{https://www.buhl.de/steuer/ratgeber/nebenkosten-abrechnung-absetzen}{buhl.de} werden von der Steuer \textbf{abgezogen}. Darunter fallen:
				\begin{itemize}
					\item Handwerker, Hausmeister, Haftpflichtversicherung
					\item Renovierungen, Reinigungen (Regenrinne, Schornsteinfeger, etc.)
					\item Winterdienst, Rauchmelder
					\item ...
				\end{itemize}\n
				\textbf{Wo \& wie angeben?} \textrightarrow\ Anlage Haushaltsnahe Aufwendungen Zeile 5\\\pause
				\textbf{Hinweis:} Tatsächliche Kosten angeben, Finanzamt betrachtet davon dann 20\%.
			\end{frame}
		
			\begin{frame}{Spenden \& Mitgliedsbeiträge}
				Spenden:
				\begin{itemize}
					\item Alles
					\begin{itemize}
						\item Solange Empfänger gemeinnützig, mildtätig oder kirchlich
						\item Empfänger mit Sitz in Deutschland
					\end{itemize}
					\item Bis 300€: Vereinfachter Nachweis (z.B. Kontoauszug)
					\item Über 300€: Spendenquittung wird ggf. nachgefordert
				\end{itemize}\n\pause
				Mitgliedsbeiträge:
				\begin{itemize}
					\item Ähnliche Regel wie oben
					\item Nicht absetzbar für Sportvereine, Heimatvereine, etc.
				\end{itemize}\n\pause
				\textbf{Wo angeben?} \textrightarrow\ Zeile 5 Anlage Sonderausgaben
			\end{frame}
			
			\begin{frame}{Was noch?}
				\begin{itemize}
					\item Umzug (beruflich motiviert): Uff, also da geht einiges ;)
					\item Häusliches Arbeitszimmer
					\item Weitere haushaltsnahe Dienstleistungen (z.B. Handwerker, Haushaltshilfe)
					\item Doppelte Haushaltsführung
					\item ...
				\end{itemize}
			\end{frame}
			
		\subsection{Sonst noch was?}
		
			\begin{frame}{Aber ich hab gar keinen Job :(}
				Du hast Kein Job? / Verdienst sehr wenig?\\\pause
				Du machst deine Zweitausbildung?\\\pause
				Trotzdem eine Steuererklärung machen!\n\pause
				\begin{itemize}
					\item Werbungskosten als Verlust ansammeln (letzte 7 Jahre)
					\item Geht nur für Zweitausbildung
					\item In Folgejahren (wenn man Steuern zahlt) anrechnen und dann sparen
				\end{itemize}\n
				\textrightarrow\ \textbf{Verlustvortrag}
			\end{frame}
		
			\begin{frame}{Schei*e, ich hab falsche Angaben gemacht!}
				\textbf{Sei ehrlich!}\n
				\begin{itemize}
					\item Versehentlich falsche Angaben zeitnah formlos korrigieren (am besten schriftlich)
					\item Nicht vorsätzlich/fahrlässig falsche Angaben \textrightarrow\ Bußgeld
					\item Vorsätzlich/Bewusst falsche Angaben gemacht \textrightarrow\ Strafverfahren wegen Steuerhinterziehung
					\begin{itemize}
						\item BGH Richtlinie: Freiheitsstrafe ab ca. 50.000€ hinterzogener Summe
					\end{itemize}
				\end{itemize}
			\end{frame}
		
		\subsection{Nochmal in Kürze}
		
			\begin{frame}
				\begin{itemize}
					\item Angestellt/Werkstudent? \textrightarrow\ freiwillige Steuererklärung\\
					Selbstständig/Freiberufler? \textrightarrow\ Pflicht zur Steuererklärung
					\item Erstausbildung? \textrightarrow\ Sonderausgaben\\
					Zweitausbildung? \textrightarrow\ Werbungskosten\\
					Freiberufler? \textrightarrow\ Betriebsausgaben\pause
					\item Steuererklärung ohne großen Aufwand online machen\pause
					\item Berufliche/Universitäre Ausgaben können Steuerlast senken
					\begin{itemize}
						\item Auch ohne Job \textrightarrow\ Verlustvortrag
					\end{itemize}
					\item Pauschalen nutzen
					\item Dokumente, Rechnungen, Quittungen, etc. aufheben
				\end{itemize}
			\end{frame}
		
			\begin{frame}
				\begin{itemize}
					\item Home-Office Pauschale: Bis 210 Tage je 6€ ($\leq$ 2022 nur 120 Tage), in Werbungskostenpauschale schon enthalten
					\item Weg zur Arbeit: 30ct / km (einfache Strecke; abgerundet)
					\item Semesterbeitrag: Fortbildungskosten
					\item Verpflegungspauschale: $>$ 8h auswärts, 14€ bzw. 28€
					\item Arbeitsmittel: 110€ Pauschale, mehr mit Rechnung
					\item Telefon \& Internet: 20\%, mehr mit Nachweisen
					\item Nebenkosten: Haushaltsnahe Dienstleistung, 20\%
					\item Spenden/Mitgliedsbeiträge: Gemeinnützig, mildtätig, kirchlich \textrightarrow\ immer Sonderausgaben
				\end{itemize}
			\end{frame}
		
			\begin{frame}
				Nützliches Material:\n
				\begin{itemize}
					\item "`Steuern mit Kopf"' \textrightarrow\ \href{https://www.youtube.com/watch?v=vEYL7AlCTgw&list=PL0OXhlRkvak8sq4efIvHYxx4e_DrE8zZT}{Playlist} Schritt für Schritt durch die Steuererklärung
					\item "`Finanzfluss"' \textrightarrow\ \href{https://www.youtube.com/watch?v=FiAGN-RrHMg}{ELSTER Tutorial} von Login bis Abschicken der Steuererklärung
					\item "`Finanztip"' \href{https://www.finanztip.de/steuererklaerung/steuererklaerung-anlage-n/}{Videos zu Anlage N}
					\item \href{https://www.steuererklaerung-verstehen.de/}{steuererklaerung-verstehen.de}
					\item \href{https://www.kann-man-das-absetzen.de/}{kann-man-das-absetzen.de}
					\item \href{https://www.buhl.de/steuer/ratgeber/}{buhl Ratgeber}
					\item \href{https://www.elster.de/eportal/helpGlobal?themaGlobal=help_est_ufa_10_2023}{ELSTER-FAQ}
				\end{itemize}
			\end{frame}
	
		{
		\setbeamertemplate{background canvas}{}
		\begin{frame}[plain]
			\begin{center}
				\includegraphics[height=\textheight]{images/too-afraid-to-ask}
			\end{center}
		\end{frame}
		}
\end{document}













