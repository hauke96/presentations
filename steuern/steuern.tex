\documentclass{beamer}
\usepackage[utf8]{inputenc}
\usepackage[OT1]{fontenc}
\usepackage[ngerman]{babel}
\usepackage{lastpage}
\usepackage{tikz}
\usepackage{lmodern}
%\usepackage{libertine}
\usepackage{tabularx}
\usepackage{hyperref}
\usepackage{xifthen}
\usepackage{multicol}
\usepackage{tikz}
\usetikzlibrary{positioning}
\usetikzlibrary{arrows}

\newcommand{\n}{\hfill\\\vspace{0.25cm}}

\renewcommand\footnoterule
{
	% Original implementation just with gray color
	\color{gray}
	\kern-3pt 
	\hrule width 2in 
	\kern 2.6pt
}
\renewcommand\thefootnote{\tiny\textcolor{gray}{\arabic{footnote}}}
\let\oldfootnote\footnote
\renewcommand{\footnote}[1]
{%
	\oldfootnote
	{
		\tiny
		\textcolor{gray}{#1}
	}%
}
\newcommand{\citewiki}[2][]
{%
	\footnote
	{
		\ifthenelse{\isempty{#1}}
		{
			Quelle: \href{https://de.wikipedia.org/wiki/#2}{Wikipedia:#2}
		}
		{
			Quelle: \href{https://de.wikipedia.org/wiki/#2}{Wikipedia:#1}
		}
	}
}
\newcommand{\citeurl}[2]
{%
	\footnote{\ Quelle: \href{#1}{#2}}
}
\newcommand{\money}[1]{
	\node
	[
		fill=green!45!gray,
		minimum height=0.4cm,
		minimum width=0.9cm,
		inner sep=0pt,
		rounded corners=0.1cm,
		draw=green!50!black,
		thick
	] at #1 {\scriptsize\color{white}€};
}
\newcommand{\smallmoney}[1]{
	\node
	[
		fill=green!45!gray,
		minimum height=0.3cm,
		minimum width=0.675cm,
		inner sep=0pt,
		rounded corners=0.075cm,
		draw=green!50!black,
		semithick
	] at #1 {\tiny\color{white}€};
}

\input{../flat-blue-theme.inc}

\setbeamercovered{invisible}
\beamertemplatenavigationsymbolsempty

\author[Hauke Stieler]{Hauke Stieler\\\href{mailto:4stieler@informatik.uni-hamburg.de}{4stieler@inf}}
\title{Steuern}
\date{\today}

\begin{document}
	{
		\setbeamertemplate{footline}{}
		\begin{frame}
			\includegraphics[width=\paperwidth,trim=2cm 2cm -1.9cm 3.35cm]{images/tax-government}
		\end{frame}
		\addtocounter{page}{-1}
	}

	{
		\setbeamertemplate{footline}{}
		\setbeamertemplate{headline}{}
%		\setbeamertemplate{background}{\includegraphics[width=\paperwidth,trim=0 0 0 3.5cm]{images/tax-government}}
		\maketitle
		\addtocounter{page}{-1}
	}
	
	\begin{frame}{Disclaimer}
		\begin{center}
			Diese Präsentation inklusive Vortrag ist keine Rechts-, Steuer- oder Finanzberatung!\n
			Es besteht keinerlei Garantie für die Richtigkeit der Informationen in dieser Präsentation, alle Angaben ohne Gewähr!\n
		\end{center}
	\end{frame}

	\begin{frame}
		\begin{center}
			Was für Vorwissen hast du?
			% TODO was hier machen? Umfrage? Welche Fragen?
		\end{center}
	\end{frame}

	\begin{frame}
		\tableofcontents[hidesubsections]
	\end{frame}
	
	\section{Basics}
	
		\begin{frame}
			\tableofcontents[currentsection,hideallsubsections]
		\end{frame}
	
		\subsection{Wieso? Weshalb? Warum?}
	
			\begin{frame}{Was sind Steuern?}
				\begin{itemize}
					\item Zahlungen an Staat/Land/Gemeinde
					\item Kein Anspruch auf Gegenleistung
					\begin{itemize}
						\item Anders als bei Abgaben, Gebühren, Maut, etc.
						\item Beispiel: Fahrräder dürfen auf Straßen fahren, obwohl es keine Fahrradsteuer gibt (sondern nur eine Kfz-Steuer)
					\end{itemize}
				\end{itemize}
			\end{frame}
		
			\begin{frame}{Warum eigentlich Steuern?}
				\textbf{Staatshaushalt} decken.
				\begin{itemize}
					\item \href{https://www.bundeshaushalt.de/DE/Bundeshaushalt-digital/bundeshaushalt-digital.html}{bundeshaushalt.de}
					\item Straßen, Eisenbahn, ÖPNV, Zuschüsse zur Rente, Bildung, BAföG, Wettervorhersage, sämtliche Ämter/Verwaltungen
				\end{itemize}
			
				\textbf{Lenkung} von Verhalten (z.B. Tabacksteuer\ \textrightarrow\ Leute sollen weniger rauchen)
				
				\textbf{Umverteilung} von reich zu arm
			\end{frame}
		
			\begin{frame}{Von wem an wen werden Steuern gezahlt?}
				Steuerzahler zahlt Steuern an Bund/Land/Gemeinde:\n
				
				\begin{description}
					\item[An Bund] Einkommenssteuer, Lohnsteuer, Umsatzsteuer
					\item[An Land] Erbschaftssteuer, Lotteriesteuer, Biersteuer
					\item[An Gemeinde] Grundsteuer, Hundesteuer
				\end{description}
			\end{frame}
		
		\subsection{Grundsätze}
	
			\begin{frame}{Maxime im Aufbau von Steuern}
				\begin{description}[labelwidth=0cm]
					\item[Gerechtigkeit] Nur wirtschaftliche Faktoren wichtig (nicht z.B. Hautfarbe)
					\item[Gleichmäßigkeit] Kein Spielraum/Willkür
					\item[Rückwirkungsverbot] Steuergesetze dürfen nicht rückwirkend in Kraft treten
					\item[Ergiebigkeit] Steuern sollten Staatshaushalt decken + keinen zu hohen Verwaltungsaufwand erzeugen
					\item[Unmerklichkeit] Steuererhebung und -belastung sollte man nicht merken
					\item[Praktikabilität] Steuergesetze sollen transparent, bestimmt und einfach sein
				\end{description}
			\end{frame}
	
	\subsection{Steuersatz}
	
		\begin{frame}
			Der Steuersatz (prozentualer Wert) kann sich wie folgt entwickeln:\n
			\begin{description}
				\item[Proportional] Immer gleicher Prozentwert (z.B. 19\% Umsatzsteuer)
				\item[Progressiv] Prozentwert steigt mit Bemessungsgrundlage (z.B. Lohnsteuer)
				\item[Regressiv] Prozentwert sinkt mit Bemessungsgrundlage \vspace{0.1cm}\newline
				{\tiny Existiert in Deutschland nicht; In USA/UK sind Sozialabgaben regressiv\\}
				\item[Stufen] Prozentwert verändert sich Stufenweise
			\end{description}
		\end{frame}
	
	\section{Glossar}
	
		\begin{frame}
			\tableofcontents[currentsection,hideallsubsections]
		\end{frame}
		
		\subsection{Glossar}
		
			\begin{frame}
				\begin{description}[labelwidth=0cm]
					\item[Steuerschuldner] Gesetzlich Verpflichtet Steuern zu zahlen
					\item[Steuerträger] Wirtschaftlich belastet\citewiki{Direkte\_und\_indirekte\_Steuer}
					\item[Steuerzahler] Person, die tatsächlich das Geld überweist\citewiki{Steuerzahler}
					\item[Veranlagung] Ermittlungsverfahren + Festsetzungsverfahren\citewiki{Veranlagung\_(Steuerrecht)}
					\item[Steuerfestsetzung] Verwaltung stellt Steuerbescheid aus\citewiki{Steuerfestsetzung}
					\item[Steuerbescheid] Zettel auf dem steht welche Steuern anfallen\citewiki{Steuerbescheid}
					\item[Bemessungsgrundlage] Wert auf dem Steuer basiert (z.B. zu versteuerndes Einkommen)\citewiki{Bemessungsgrundlage\_(Steuerrecht)}
				\end{description}
			\end{frame}
	
	\section{Steuerarten}
	
		\begin{frame}
			\tableofcontents[currentsection,hideothersubsections]
		\end{frame}
	
		\subsection{Direkte / indirekte Steuern}
		
			\begin{frame}
				\begin{tabularx}{\linewidth}{X|X}
					\multicolumn{1}{c|}{\textbf{Direkt}} &
					\multicolumn{1}{c}{\textbf{Indirekt}} \\[0.25cm]
					Schuldner = Träger & Schuldner $\neq$ Träger \\
					\vspace{0.25cm}Beispiel Lohnsteuer: \newline
						Ich (Schuldner) muss von meinem Lohn Steuer \textbf{direkt} ans Finanzamt zahlen. Meist vom Arbeitgeber übernommen, es ist aber \textbf{mein Geld}, das überwiesen wird, ich (Träger) trage die Steuerlast selbst. & 
					\vspace{0.25cm}Beispiel Mehrwertsteuer: \newline
						Kunden (Schuldner) zahlen Steuern \textbf{indirekt}, da Verkäufer (Träger) diese auf \textbf{seine Einnahmen} zahlen muss\ \textrightarrow\ Steuer ist daher im Preis mit enthalten = Kunde \textbf{trägt} die Steuerlast.
				\end{tabularx}\\
				\citewiki{Direkte\_und\_indirekte\_Steuer}
			\end{frame}
		
		\subsection{Personen- / Realsteuer}
		
			\begin{frame}
				\begin{tabularx}{\linewidth}{X|X}
					\multicolumn{1}{c|}{\textbf{Personensteuer}\citewiki{Personensteuer}} &
					\multicolumn{1}{c}{\textbf{Realsteuer}\citewiki{Realsteuer}} \\[0.25cm]
					Steuer \textbf{abhängig} von persönlichen Umständen (Alter, Familie, etc.). & Steuer \textbf{unabhängig} von Personen. \\
					\vspace{0.25cm} Beispiel: Lohnsteuer &
					\vspace{0.25cm} Beispiel: Grundsteuer
				\end{tabularx}
			\end{frame}
		
		\subsection{Quellen- / Veranlagungssteuer}
		
			\begin{frame}
				\begin{tabularx}{\linewidth}{X|X}
					\multicolumn{1}{c|}{\textbf{Quellensteuer}\citewiki{Quellensteuer}} &
					\multicolumn{1}{c}{\textbf{Veranlagungssteuer}\citeurl{https://www.steuererklaerung-verstehen.de/lexikon/veranlagungssteuer}{steuererklaerung-verstehen.de}} \\[0.25cm]
					Steuer wird sofort direkt an Quelle erhoben. & Steuer wird zu anderem Zeitpunkt (z.B. Steuererklärung im Folgejahr) erhoben.\\
					\vspace{0.25cm} Beispiel Lohnsteuer:\newline
						Arbeitgeber (Quelle) überweist mir meinen Lohn und meine Lohnsteuer ans Finanzamt. &
					\vspace{0.25cm} Beispiel Umsatzsteuer:\newline
						Umsatzsteuer wird im Voraus entrichtet, nicht erst, wenn Einnahmen entstehen. Daher ist nach Jahresende eine Steuererklärung Pflicht.
				\end{tabularx}
			\end{frame}
		
		\subsection{Pauschal- / Individualsteuer}
		
			\begin{frame}
				\begin{tabularx}{\linewidth}{X|X}
					\multicolumn{1}{c|}{\textbf{Pauschalsteuer}} &
					\multicolumn{1}{c}{\textbf{Individualsteuer}} \\[0.25cm]
					Steuersatz (Prozent-Zahl) immer gleich. & Steuersatz individuell von persönlichen Verhältnissen.\\
					\vspace{0.25cm} Beispiel Umsatzsteuer:\newline
						Immer 7\% bzw. 19\%. &
					\vspace{0.25cm} Beispiel Lohnsteuer:\newline
						Steuersatz abhängig von Gehalt.
				\end{tabularx}
			\end{frame}
	
	\section{Steuererklärung}
	
		\begin{frame}
			\tableofcontents[currentsection,hideallsubsections]
		\end{frame}
	
	{
		\setbeamertemplate{background canvas}{}
		\begin{frame}[plain]
			\begin{center}
				\includegraphics[height=\textheight]{images/too-afraid-to-ask.jpg}
			\end{center}
		\end{frame}
	}
\end{document}













