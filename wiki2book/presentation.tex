\documentclass[9pt]{beamer}
\input{../flat-blue-theme.inc}

\usepackage[utf8]{inputenc}
\usepackage[OT1]{fontenc}
\usepackage[ngerman]{babel}
\usepackage{lastpage}
\usepackage{xcolor}
\usepackage{lmodern}
\usepackage{tabularx}

\usepackage{tikz}
\usetikzlibrary{positioning}
\usetikzlibrary{arrows}
\usetikzlibrary{fit}

\setbeamercovered{invisible}

\author[Hauke Stieler]{Hauke Stieler}
\title{wiki2book}
\subtitle{Aus Wikipedia eigene eBooks bauen}
\date{\today}

\begin{document}
	{
		\setbeamertemplate{headline}{}
%		\setbeamertemplate{background}{\includegraphics[width=\paperwidth,trim=0 0 0 3.5cm]{images/title}}
		\maketitle
	}

	% Kurze Infos zu mir?

	% Motivation
	% - Abstieg in immer tiefere Wikipedia Artikel
	% - Warum nicht tool XY?
	\section{Motivation}	
	
	\subsection{Wikipedia hole}
	
	\begin{frame}
		\begin{center}
			Kennt ihr das?
		\end{center}
	\end{frame}

	\begin{frame}
		\begin{center}
			\begin{tikzpicture}
				\visible<1->{\node at (0,0)			{\includegraphics[height=0.75\textheight]{images/wikipedia-screenshot-0.jpg}};}
				\visible<2->{\node at (0.25, -0.25)	{\includegraphics[height=0.75\textheight]{images/wikipedia-screenshot-1.jpg}};}
				\visible<3->{\node at (0.5, -0.5)	{\includegraphics[height=0.75\textheight]{images/wikipedia-screenshot-2.jpg}};}
				\visible<4->{\node at (0.75, -0.75)	{\includegraphics[height=0.75\textheight]{images/wikipedia-screenshot-3.jpg}};}
				\visible<5->{\node at (1, -1)		{\includegraphics[height=0.75\textheight]{images/wikipedia-screenshot-4.jpg}};}
				\visible<6->{\node at (1.25, -1.25)	{\includegraphics[height=0.75\textheight]{images/wikipedia-screenshot-5.jpg}};}
			\end{tikzpicture}
		\end{center}
	\end{frame}

	\begin{frame}
		\textit{"`Going on to Wikipedia to look something up, then unexpectedly being sucked into a seemingly \textbf{endless series of link clicking} to end up in a completely different part of wikipedia than you ever meant to go to."'}
		\n\hfill{\small--- \href{https://www.urbandictionary.com/define.php?term=Wikipedia\%20Hole}{Urban Dictionary}}
	\end{frame}

	\begin{frame}
		\begin{center}
			\only<1>{\includegraphics[height=0.75\textheight]{images/meme-spongebob-gary-0.png}}
			\only<2>{\includegraphics[height=0.75\textheight]{images/meme-spongebob-gary-1.png}}
		\end{center}
	\end{frame}

	\subsection{Existierende Tools}
	
	\begin{frame}
		\begin{itemize}
			\item \texttt{pandoc}
			\item \texttt{mediawiki2latex} / \texttt{wb2pdf}
			\item \texttt{epub-press}
			\item \texttt{w2eb}
			\item \texttt{percollate}
		\end{itemize}
	\end{frame}

	\begin{frame}{Warum gehen die nicht?}
		Inhaltliche \& visuelle Gründe:
		\begin{itemize}
			\item Formatierung, Schriftgrößen, etc. stimmt nicht
			\item Templates werden nicht/uneingeschränkt evaluiert
			\item \LaTeX/Math wird nicht in Bild gerendert
			\item Tabellen funktionieren nicht
		\end{itemize}
	\end{frame}

	\begin{frame}{Warum gehen die nicht?}
		Technische Gründe:
		\begin{itemize}
			\item Kann nicht mehrere Artikel gleichzeitig
			\item Bilder werden nicht heruntergeladen
			\item Wird nicht mehr maintained
			\item Ist in JavaScript
			\item Ist in einer Programmiersprache, die ich nicht kann / mag
			\item Ergebnis ist kein EPUB
			\item Ergebnis lief nicht auf meinem Tolino
		\end{itemize}
	\end{frame}
	
	% Was will ich haben? Was soll das Resultat sein?
	\subsection{Was will ich haben?}
	\begin{frame}
		\begin{center}
			\textbf{Generierte und gekaufte eBooks sollen sich qualitativ nicht unterscheiden.}
		\end{center}\pause
		Allgemeine Anforderungen:
		\begin{itemize}
			\item Formatierung stimmig
			\item Korrekte Übersetzung/Einbindung von Tabellen, Bilder, Listen, Quellenangaben, etc.
			\item Wikipedia-spezifische Templates \& Kategorien ignorieren
		\end{itemize}\pause\n
		Persönliche Anforderungen:
		\begin{itemize}
			\item Soll auf meinem Tolino eBook-Reader laufen
			\item Go als Programmiersprache
			\item Caching aller heruntergeladenen Daten (zum Coden im Zug)
		\end{itemize}
	\end{frame}

	\section{Wikipedia}	
	
	% Mediawiki Sprache
	\subsection{Wikitext -- Die Sprache der Wikipedia}
	
	\begin{frame}
		\vspace{1cm}
		\begin{center}
			Wie Wikipedia funktioniert
		\end{center}
	\end{frame}
	
	\begin{frame}[fragile]{Formatierung}
		\begin{verbatim}
Wikitext ''kann'' auch '''Formatierung'''.

Und  '''sogar ''alles''' durcheinander'' geht.
		\end{verbatim}
		\hrule\n
		\includegraphics[scale=0.35]{images/wikitext-example-0-formatting.png}
	\end{frame}
	
	\begin{frame}[fragile]{Links}
		\begin{verbatim}
Interne [[Hyperlink|Links]] gehen.

Auch ins Internetz [https://externe-links], sogar mit
[https://foo.bar Namen].
		\end{verbatim}
		\hrule\n
		\includegraphics[scale=0.35]{images/wikitext-example-1-links.png}
	\end{frame}
	
	\begin{frame}[fragile]{Referenzen \& Templates}
		\begin{verbatim}
Hi<ref name="foo">{{Internetquelle|url=http://bar.de
|abruf=2022-10-12|titel=Ref mit Template}}</ref>!

Die selbe Ref. nochmal!<ref name="foo" />
		\end{verbatim}
		\hrule\n
		\includegraphics[scale=0.35]{images/wikitext-example-2-refs.png}
	\end{frame}
	
	\begin{frame}[fragile]{Überschriften}
		\begin{verbatim}
= Level 1 =
Wird nicht aktiv benutzt, da Titel der Seite h1 ist.

==== Level 4 ====
Die hier wird benutzt.
		\end{verbatim}
		\hrule\n
		\includegraphics[scale=0.35]{images/wikitext-example-3-headings.png}
	\end{frame}

	\begin{frame}[fragile]{Listen}
		\begin{verbatim}
* Listen
** gibt
es
# auch
## noch
		\end{verbatim}
		\hrule\n
		\includegraphics[scale=0.35]{images/wikitext-example-4-lists.png}
	\end{frame}

	\begin{frame}[fragile]{Tabellen}
		\footnotesize
		\begin{verbatim}
{| class="wikitable"
|-
! Spalte 1 !! Spalte 2
|-
| Hier
| könnte
|-
| ihre || Werbung stehen
|}
		\end{verbatim}
		\hrule\n
		\includegraphics[scale=0.35]{images/wikitext-example-5-table.png}
	\end{frame}

	\begin{frame}[fragile]{Bilder}
		\begin{verbatim}
Hier ein Bild:

[[Datei:Full moon partially obscured by atmosphere.jpg
|mini|Mit Unterschrift.]]
		\end{verbatim}
		\hrule\n
		\includegraphics[scale=0.35]{images/wikitext-example-6-image.png}
	\end{frame}
	
	\begin{frame}{Und vieles mehr}
		\begin{itemize}
			\item Description list
			\item Zitate
			\item Einrückungen
			\item Code
			\item \LaTeX-Mathe-Zeug
			\item Musiknoten
			\item Gallerien
			\item Inline Bilder
			\item Diverse Parameter an allen möglichen Dingen
		\end{itemize}
	\end{frame}
	
	% Wie funktioniert Wikipedia?
	% - Aufbau (einzelne Instanzen, Wikimedia)
	% - Benötigte APIs und wie diese funktionieren
	\subsection{Instanzen \& APIs}
	
	\begin{frame}{Instanzen -- Artikel}
		\begin{itemize}
			\item Instanz pro Land/Sprache \textrightarrow\ z.B. [en\textbar de\textbar nds].wikipedia.org
			\item Verlinkungen ggf. zu anderen Instanzen möglich
		\end{itemize}
	\end{frame}

	\begin{frame}{Instanzen -- Bilder}
		\begin{itemize}
			\item Wikimedia commons (\href{https://commons.wikimedia.org}{commons.wikimedia.org})
			\item Normal:\newline \href{https://upload.wikimedia.org/wikipedia/commons/0/06/Foo.jpg}{upload.wikimedia.org/wikipedia/commons/0/06/Foo.jpg}
			\item Aber auch:\newline \href{https://upload.wikimedia.org/wikipedia/de/2/26/Son-3.jpg}{upload.wikimedia.org/wikipedia/\textbf{de}/2/26/Son-3.jpg}
			\pause
			
			\item Redirects möglich
			\begin{itemize}
				\item Beispiel: \href{https://commons.wikimedia.org/w/index.php?title=File:MET00506.jpg&action=edit}{\texttt{File:MET00506.jpg}}
				\item Ggf. ist Dateiname im Artikel $\neq$ Dateiname bei Wikimedia commons
				\item Nach Bild-Artikel abfragen
				\item \texttt{redirects=true} Parameter hilft
			\end{itemize}
			\pause
			
			\item In Deutschen Artikeln wird natürlich \texttt{\textbf{Datei}:Sol-3.jpg} benutzt
		\end{itemize}
	\end{frame}

	\begin{frame}{APIs -- Bilder}
		\textbf{Aufbau:}\\
		\texttt{upload.wikimedia.org/wikipedia/\{instance\}/}
		\texttt{\{MD5[0]\}/\{MD5[0]MD5[1]\}/\{filename\}}
		\n
		\textbf{MD5:}\\
		\texttt{MD5[i]} = Das \texttt{i}-te Zeichen des MD5-Hashes von \texttt{filename}
	\end{frame}

	\begin{frame}[fragile]{APIs -- Artikel abfragen}
		\textbf{Anfrage:}\\
		Puren Wikitext in JSON Antwort verpackt:\\
		\begin{verbatim}
GET  de.wikipedia.org/w/api.php
       ?action=parse
       &format=json
       &prop=wikitext
       &page={article name}
		\end{verbatim}\n
		\textbf{Antwort:}
		\begin{verbatim}
{
    "parse": {
        "title": "Erde",
        "wikitext": {
            "*": "..."
        }
    }
}
		\end{verbatim}
	\end{frame}

	\begin{frame}[fragile]{APIs -- Templates evaluieren}
		\textbf{Anfrage:}\\
		Wie bei Artikeln nur andere Parameter.\\
		\begin{verbatim}
GET  de.wikipedia.org/w/api.php
        ?action=expandtemplates
        &format=json
        &prop=wikitext
        &text={{mein tolles template}}
		\end{verbatim}\n
		\textbf{Antwort:}
		\begin{verbatim}
{
    "expandtemplates": {
        "wikitext": "..."
    }
}
		\end{verbatim}
	\end{frame}

	\begin{frame}{APIs -- \LaTeX-Mathe in Bild umwandeln}
		\begin{enumerate}
			\item Math-check API für Resource location anfragen
			\item Eigentliches Bild abfragen
		\end{enumerate}
	\end{frame}

	\begin{frame}[fragile]{\LaTeX zu Bild: 1. Resource location bekommen}
		\textbf{Anfrage:}\\
		URL: \texttt{POST https://wikimedia.org/api/rest\_v1/media/math/check/tex}\\
		Body: URL encoded form Element \texttt{q} mit dem \LaTeX-Code:\n
		\verb+q:\sqrt{x}+
		\n
		\textbf{Antwort:}\\
		Header \texttt{x-resource-location} auslesen:\n
		\texttt{x-resource-location: 73b85c4ec364802ad746381712d10a43f073d50a}
	\end{frame}

	\begin{frame}[fragile]{\LaTeX zu Bild: 2. Bild abfragen}
		\textbf{Anfrage:}\\
		Einfaches \texttt{GET} mit Hash an\n
		\href{https://wikimedia.org/api/rest_v1/media/math/render/svg/73b85c4ec364802ad746381712d10a43f073d50a}{\texttt{wikimedia.org/api/rest\_v1/media/math/render/\{svg\textbar png\}/73b85c4...}}
	\end{frame}
		
	% Technische Strategie
	% - Umwandlung mediawiki -> HTML (weil EPUB HTML enthält) -> EPUB
	% - mediawiki -> HTML = Quasi Transpiler
	% - Wie funktioniert ein Compiler/Transpiler
	\section{Technischer Aufbau}
	
	\begin{frame}
		\vspace{1cm}
		\begin{center}
			Wie wiki2book funktioniert
		\end{center}
	\end{frame}
	
	\subsection{Technischer Aufbau}
	
	\begin{frame}{Idee \& Annahmen}
		Annahmen:
		\begin{itemize}
			\item Heruntergeladener wikitext ist korrekt \textrightarrow\ Keine Syntaxprüfung nötig.
			\item Formatierung vom HTML (Einrückung, Leerzeilen, etc.) ist egal
		\end{itemize}\pause
		\vspace{0.25cm}
		Warum HTML?\\
		EPUB eBooks sind quasi zezippte Webseiten.\pause\n
		Idee:
		\begin{itemize}
			\item Elemente im wikitext rekursiv durch Token ersetzen
			\item Token in Map speichern: \texttt{[token]} \textrightarrow\ \texttt{[token-content]}
			\item Token-Content kann weitere Token enthalten
			\item Token sollen einfach zu HTML ersetzt/relaxiert werden können
		\end{itemize}
	\end{frame}
	
	\begin{frame}{Meine "`Compiler"' "`Pipeline"'}
		\begin{tikzpicture}[node distance=3cm]
			\visible<1->
			{
				\node (wikitext) at (0,0) {\texttt{wikitext}};
				\node[align=left,font=\scriptsize\ttfamily,anchor=north] (wikitext-example) [below=1cm of wikitext.center]
					{Die '''Erde''' ist\\ein [[Planet]] ...};
			}
			
			\visible<2->
			{
				\node (parser) [right of=wikitext] {Token/AST};
				\node[align=left,font=\scriptsize\ttfamily,anchor=north] (parser-example) [below=1cm of parser.center]
					{Die \$\$TOK\_BOLD\_1\$\$ ist\\ein \$\$TOK\_LINK\_2\$\$ ...\\~\\~\\{\normalfont Token Map:}\\\$\$TOK\_BOLD\_1\$\$ \textrightarrow\ Erde\\\$\$TOK\_LINK\_2\$\$ \textrightarrow\ Planet};
				\draw[->,->=stealth'] (wikitext) -- (parser);
			}
			
			\visible<3->
			{
				\node (html) [right of=parser] {\texttt{HTML}};
				\node[align=left,font=\scriptsize\ttfamily,anchor=north] (html-example) [below=1cm of html.center]
				{buch.html\\~\\Die <b>Erde</b>\\ist ein <a href=\\\dq...\dq>Planet</a>\\...};
				\draw[->,->=stealth'] (parser) -- (html);
			}
			
			\visible<4-5>
			{
				\node (epub) [right of=html] {\texttt{EPUB}};
				\draw[->,->=stealth'] (html) -- (epub);
				\node[align=left,font=\scriptsize\ttfamily,anchor=north] (epub-example) [below=1cm of epub.center]
				{erde.epub};
			}
		
			\visible<6->
			{
				\node (html2) [right=0.75cm of html] {\texttt{HTML}};
				\node (epub) [right of=html] {\texttt{EPUB}};
				\draw[->,->=stealth'] (html) -- (html2);
				\draw[->,->=stealth'] (html2) -- (epub);
				\node[align=center,font=\scriptsize,anchor=north] (html2-example) [below=1cm of html2.center]
				{what\\the\\...};
				\node[align=left,font=\scriptsize\ttfamily,anchor=north] (epub-example) [below=1cm of epub.center]
				{buch.epub};
			}
			
			\visible<5->
			{
				\draw (wikitext.west)++(0,0.5) -- ++(0,0.5) -- node[above] {wiki2book} ([xshift=-0.05cm]html |- 0,1) -- ++(0,-0.5);
				\draw ([xshift=0.05cm]html |- 0,1)++(0,-0.5) -- ++(0,0.5) -- node[above] {pandoc} (epub.east |- 0,1) -- ++(0,-0.5);
			}
		
			\visible<7->
			{
				\node[yshift=-0.75cm] at (html2) {\includegraphics[width=4.5cm]{images/meme-spongebob-pandoc-html-fire.png}};
			}
		\end{tikzpicture}
	\end{frame}
	
	% Aufbau von wiki2book
	% - Einzelne Schritte:
	%  1. Clean
	%  2. Templates evaluieren
	%  3. Allerlei Dinge (Links, Bilder, Tabellen, Listen, ...)
	%  Solange zu 1 zurück springen bis alles fertig ist
	\section{Funktionsweise}
	
	\begin{frame}
		\vspace{1cm}
		\begin{center}
			Interner Aufbau
		\end{center}
	\end{frame}
	
	\subsection{Ablauf}
	
	\begin{frame}
		\begin{center}
			\vspace{-0.65cm}
			\begin{tikzpicture}[node distance=1.15cm]
				\node (load) at (0,0) {1. Load wikitext};
				
				\visible<2->
				{
					\node (clean) [below of=load] {2. Clean data};
					\draw[->,->=stealth'] (load) -- (clean);
				}
				
				\visible<3->
				{
					\node (evaluate) [below of=clean] {3. Evaluate templates};
					\draw[->,->=stealth'] (clean) -- (evaluate);
				}
				
				\visible<4->
				{
					\node (parse) [below of=evaluate] {4. Parse data};
					\draw[->,->=stealth'] (evaluate) -- (parse);
				}
				
				\visible<5->
				{
					\draw[->,->=stealth'] (parse) to [out=0,in=0,looseness=1.75] node (if-no-new-token) [right] {\textit{if new token created}} (clean);
				}
				
				\visible<6->
				{
					\node (generate) [below of=parse] {5. Generate HTML};
					\draw[->,->=stealth'] (parse) -- node [right] {\textit{if \textbf{no} new token created}} (generate);
					\node[draw,label=90:wiki2book,fit=(load) (clean) (evaluate) (parse) (generate) (if-no-new-token)] {};
				}
				
				\visible<7->
				{
					\node (pandoc) [below right=0.75cm and 0.25cm of generate] {pandoc};
					\draw[->,->=stealth'] (generate) to [out=270,in=180] (pandoc);
				}
				
				\visible<8->
				{
					\node (ebook) [right=1.25cm of pandoc] {\texttt{book.epub}};
					\draw[->,->=stealth'] (pandoc) -- (ebook);
				}
			\end{tikzpicture}
		\end{center}
	\end{frame}

	\subsection{Pipeline}
	
	\begin{frame}{1. Artikel laden}
		Via CLI über drei mögliche Wege:
		\begin{itemize}
			\item Aus Datei (\texttt{standalone <file>})
			\item Einzelnen Artikel (\texttt{article <name>})
			\item Projekt-Datei (\texttt{project <project-file>})
		\end{itemize}
	\end{frame}

	\begin{frame}{2. Cleanup}
		\begin{itemize}
			\item Entfernt Kategorien
			\item Entfernt unerwünschte Templates
			\item Entfernt HTML (\texttt{div} und \texttt{span} Elemente)
			\item Entfernt leere Abschnitte
		\end{itemize}
	\end{frame}

%	\begin{frame}{2. Cleanup -- Irgendwas mit Performance oder so}
%		\begin{center}
%			\includegraphics[width=\linewidth]{images/code-cleanup.png}
%		\end{center}
%	\end{frame}

	\begin{frame}{3. Templates evaluieren}
		\begin{itemize}
			\item Finde templates
			\item Bitte Wikipedia-API diese zu wikitext zu konvertieren
			\item String-Replace
		\end{itemize}
		~\\
		\pause
		\begin{center}
			\includegraphics[height=0.5\textheight]{images/meme-spongebob-duh.png}
		\end{center}
	\end{frame}

	\begin{frame}[fragile]{4. Parsen}
		\textit{Kann man nicht einfach mit Regexes lösen?}\\\pause
		Haha.\pause\ Nein.\pause..\ Manchmal.
	\end{frame}

	\begin{frame}[fragile]{4. Parsen -- Beispiel: Referenzen}
		Wikitext:\\
		~~~~\verb+Foo.<ref name="foo">...</ref> Bar.<ref name="foo"/>+\n\pause
		Regex für ref-Definition:\\
		~~~~\verb+<ref[^>]*?name="?([^"^>]*)"?([^>]*?=[^>]*?)* ?>((.|\n)*?)</ref>+\n
		Regex für ref-Nutzung:\\
		~~~~\verb+<ref name="(.*?)"\s?/>+
	\end{frame}

	\begin{frame}[fragile]{4. Parsen -- Beispiel: \textbf{bold} und \textit{italic}}
		Wikitext:\\
		~~~~\verb+''foo'' '''bar'''+ \textrightarrow\ \textit{foo} \textbf{bar}\\\pause
		~~~~\verb+''''foo+\pause\verb+'''+\pause \textrightarrow\ '\textbf{foo}\n\pause
		Kein Regex möglich, da \textit{kontextsensitive} Grammatik\footnote{... glaube ich} nötig wäre.\n
		Entscheidbarkeit bei kontextsensitiven Grammatiken in $\mathcal{O}(2^n)$.\pause\ Laufzeit meines Algorithmus: $\mathcal{O}(2^n)$.\pause.. Yeii...
	\end{frame}

	\begin{frame}[t,fragile]{4. Parsen -- Beispiel: Tabellen}
		Wikitext (vor Preprocessing):\\
		\begin{verbatim}
{| class="wikitable"
|+ Text der Überschrift
|-
! Überschrift 1 !! Überschrift 2
|-
|style="text-align:center"|Foo || Bar
|}

		\end{verbatim}
		\textrightarrow\ Preprocessing kann Dinge vereinfachen
	\end{frame}

	\begin{frame}[t,fragile]{4. Parsen -- Beispiel: Tabellen}
		Wikitext (nach Preprocessing):\\
		\begin{verbatim}
{| class="wikitable"
|+ Text der Überschrift
|-
! Überschrift 1
! Überschrift 2
|-
|style="text-align:center"|Foo
| Bar
|}
		\end{verbatim}
		\pause
		\begin{itemize}
			\item Zeilenpräfix anschauen
			\item Styles beachten
			\item Tabellen in Tabellen dann auch easy
		\end{itemize}
	\end{frame}

	\subsection{Token-Map}

	\begin{frame}[fragile]{Beispiel}
		\textbf{Wikitext:} (vorher)\\
		\begin{verbatim}
Die Sonne ist ein:
* [https://de.wikipedia.org/wiki/Stern Stern]
		\end{verbatim}\n\pause
		\textbf{Wikitext:} (nachher)\\
		\begin{verbatim}
Die Sonne ist ein:
$$TOKEN_UNORDERED_LIST_4$$
		\end{verbatim}\n\pause
		\textbf{Token-Map:}\n
		\hspace{-0.65cm}
		\addtolength{\tabcolsep}{-0.5\tabcolsep}
		\begin{tabularx}{\textwidth}{rcp{6cm}}
			\verb+$$TOKEN_UNORDERED_LIST_4$$+		& \textrightarrow & \verb+$$TOKEN_LIST_ITEM_3$$+\\
			\verb+$$TOKEN_LIST_ITEM_3$$+			& \textrightarrow & \verb+$$TOKEN_EXTERNAL_LINK_2$$+\\
			\verb+$$TOKEN_EXTERNAL_LINK_2$$+		& \textrightarrow & \verb+$$TOKEN_EXTERNAL_LINK_URL_0$$+\newline\verb+$$TOKEN_EXTERNAL_LINK_TEXT_1$$+\\
			\verb+$$TOKEN_EXTERNAL_LINK_URL_0$$+	& \textrightarrow & \verb+https://de.wikipedia.org/wiki/Stern+\\
			\verb+$$TOKEN_EXTERNAL_LINK_TEXT_1$$+	& \textrightarrow & Stern\\
		\end{tabularx}
	\end{frame}

	\subsection{HTML Generierung}

	\begin{frame}[fragile]
		\textbf{Token-Map:}\n
		\hspace{-0.65cm}
		\addtolength{\tabcolsep}{-0.5\tabcolsep}
		\begin{tabularx}{\textwidth}{rcp{6cm}}
			\verb+$$TOKEN_EXTERNAL_LINK_2$$+		& \textrightarrow & \verb+$$TOKEN_EXTERNAL_LINK_URL_0$$+\newline\verb+$$TOKEN_EXTERNAL_LINK_TEXT_1$$+\\
			\verb+$$TOKEN_EXTERNAL_LINK_URL_0$$+	& \textrightarrow & \verb+https://de.wikipedia.org/wiki/Stern+\\
			\verb+$$TOKEN_EXTERNAL_LINK_TEXT_1$$+	& \textrightarrow & Stern\\
		\end{tabularx}
		~\n\pause
		\textbf{Template ausfüllen:}\n
		\begin{onlyenv}<2>
			\verb+<a href="              %s                   ">  %s </a>+
			\verb+<a href="https://de.wikipedia.org/wiki/Stern">Stern</a>+
		\end{onlyenv}
		\visible<3>
		{
			\includegraphics[width=\linewidth]{images/code-html-generation.png}
		}
	\end{frame}
	
	\section{Take aways}
	
	\begin{frame}
		\vspace{1cm}
		\begin{center}
			Sonst noch was?
		\end{center}
	\end{frame}

	% Probleme, die aufgetreten sind
	\begin{frame}{Sonstige Problemchen}
		\begin{itemize}
			\item Tolino ist weird
			\begin{itemize}
				\item Absturz bei Nutzung von CSS \texttt{gap} Eigenschaft
				\item Absturz bei anderen Kleinigkeiten
			\end{itemize}\pause
			\item Pandoc ist weird
			\begin{itemize}
				\item Baut HTML um \& ergänzt eigenmächtig \texttt{<p>} Elemente
				\item \texttt{<th>} in \texttt{<tr>} werden umgewandelt
			\end{itemize}\pause
			\item Wikipedia-Artikel sind weird/ungünstig
			\begin{itemize}
				\item Random Leerzeilen
				\item Uneinheitliche Template-Nutzungen
				\item Verweise auf andere Artikel (ala "`Siehe hierzu Artikel [[foo]]."')
			\end{itemize}
		\end{itemize}
	\end{frame}
	
	% Infos (Repo, Bauen, CLI, ...)
	\begin{frame}
		\vspace{1cm}
		\centering\href{https://github.com/hauke96/wiki2book}{github.com/hauke96/wiki2book}
	\end{frame}
	
	% Hands-on
	\begin{frame}
		\vspace{1cm}
		\centering Hands-on
	\end{frame}
\end{document}