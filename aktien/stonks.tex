\documentclass{beamer}
\usepackage[utf8]{inputenc}
\usepackage[OT1]{fontenc}
\usepackage[ngerman]{babel}
\usepackage{lastpage}
\usepackage{tikz}
\usepackage{lmodern}
%\usepackage{libertine}
\usepackage{tabularx}
\usepackage{hyperref}
\usepackage{xifthen}
\usepackage{multicol}
\usepackage{tikz}
\usetikzlibrary{positioning}
\usetikzlibrary{arrows}

\newcommand{\n}{\hfill\\\vspace{0.25cm}}

\renewcommand\footnoterule
{
	% Original implementation just with gray color
	\color{gray}
	\kern-3pt 
	\hrule width 2in 
	\kern 2.6pt
}
\renewcommand\thefootnote{\tiny\textcolor{gray}{\arabic{footnote}}}
\let\oldfootnote\footnote
\renewcommand{\footnote}[1]
{%
	\oldfootnote
	{
		\tiny
		\textcolor{gray}{#1}
	}%
}
\newcommand{\citewiki}[2][]
{%
	\footnote
	{
		\ifthenelse{\isempty{#1}}
		{
			Quelle: \href{https://de.wikipedia.org/wiki/#2}{Wikipedia:#2}
		}
		{
			Quelle: \href{https://de.wikipedia.org/wiki/#2}{Wikipedia:#1}
		}
	}
}
\newcommand{\citeurl}[2]
{
	\footnote
	{
		\tiny
		\textcolor{gray}
		{
			Quelle: \href{#1}{#2}
		}
	}
}
\newcommand{\money}[1]{
	\node
	[
		fill=green!45!gray,
		minimum height=0.4cm,
		minimum width=0.9cm,
		inner sep=0pt,
		rounded corners=0.1cm,
		draw=green!50!black,
		thick
	] at #1 {\scriptsize\color{white}€};
}
\newcommand{\smallmoney}[1]{
	\node
	[
		fill=green!45!gray,
		minimum height=0.3cm,
		minimum width=0.675cm,
		inner sep=0pt,
		rounded corners=0.075cm,
		draw=green!50!black,
		semithick
	] at #1 {\tiny\color{white}€};
}

\input{flat-blue-theme.inc}

\setbeamercovered{invisible}

\author[Hauke Stieler]{Hauke Stieler\\\href{mailto:4stieler@informatik.uni-hamburg.de}{4stieler@inf}}
\title{Stonks}
\date{\today}

\begin{document}
	{
		\setbeamertemplate{headline}{}
		\setbeamertemplate{background}{\includegraphics[width=\paperwidth,trim=0 0 0 3.5cm]{images/stonks-title-full}}
		\maketitle
	}
	
	\begin{frame}{Disclaimer}
		\begin{center}
			Diese Präsentation inklusive Vortrag ist keine Rechts-, Steuer-, Anlage- oder Finanzberatung!\n
			Der Handel mit Finanzprodukten ist mit Risiken behaftet und kann zum TOTALVERLUST des eingesetzten Kapitals führen.\n
			Es besteht keinerlei Garantie für die Richtigkeit der Informationen in dieser Präsentation.\n
			Alle genannten Finanzprodukte dienen nur als anschauliche Beispiele und stellen keinerlei Empfehlung dar.
		\end{center}
	\end{frame}

	\begin{frame}
		\begin{center}
			Was für Vorwissen hast du?
		\end{center}
	\end{frame}

	\begin{frame}
		\tableofcontents[hidesubsections]
	\end{frame}
	
	\section{Basics}
	
		\begin{frame}
			\tableofcontents[currentsection,hideallsubsections]
		\end{frame}
	
		\subsection{Begrifflichkeiten}
		
		\begin{frame}
			\begin{description}[labelwidth=0cm]
				\item[Gewinn/Verlust] Differenz Kauf- \& Verkaufspreis\pause
				\item[Kurs] Aktueller Preis einer Aktie an einer bestimmten Börse\pause
				\item[Rendite] Gewinn relativ zum Einsatz (in Prozent)\pause
				\item[Kapital] = Mittel = Geld (meistens)\pause
				\item[Anleger] Auch \textit{Investor}, nimmt am Finanzmarkt teil, z.B. durch Investitionen\pause
				\item[Risiko] Wahrscheinlichkeit des Eintretens eines (meist negativen) Ereignisses\pause
				\item[Portfolio] Sammlung an Wertpapieren
			\end{description}
		\end{frame}
		
		\subsection{Investieren}
		
			\begin{frame}
				\begin{definition}
					Einsatz von Kapital für einen bestimmten Verwendungszweck.\citewiki{Investition}
				\end{definition}\hfill
			
				Beispiele:
				\begin{itemize}
					\item Kauf von Aktien zum Vermögensaufbau.
					\item Kauf eines Buches zum lernen für eine Uni-Veranstaltung.
					\item Sport treiben zum Erhalt der körperlichen Gesundheit.
				\end{itemize}\hfill
			
				Horizont: Langfristig
			\end{frame}
		
		\subsection{Spekulieren}
		
			\begin{frame}
				\begin{definition}
					Risikoreiches Ausnutzen von Preisschwankungen zur Gewinnmitnahme.\citewiki{Spekulation\_(Wirtschaft)}
				\end{definition}\hfill
			
				Beispiele:
				\begin{itemize}
					\item Kauf einer Aktie, Verkauf am Folgetag: Spekulation auf steigende Kurse.
					\item Mit 100km/h auf eine rote Ampel zufahren: Spekulation, dass Ampel gleich grün wird.
				\end{itemize}\hfill
				
				Horizont: Kurzfristig
			\end{frame}
		
		\subsection{Sparen}
		
			\begin{frame}
				\begin{definition}
					Verzicht auf Verbrauch von Kapital zur späteren Verwendung.\citewiki{Sparen}
				\end{definition}\hfill
				
				Beispiele:
				\begin{itemize}
					\item Monatlich 300€ nicht ausgeben
					\item Einmalig 10.000€ auf ein Festgeldkonto
					\item Bis zum 18. Geburtstag des Kindes jährlich 500€ kontinuierlich investieren
				\end{itemize}
			\end{frame}
		
		\subsection{Zins und Zinseszins}
		
			\begin{frame}{Begrifflichkeiten}
				\begin{description}[labelwidth=0cm]
					\item[Gläubiger] Kredit\textit{geber}, vergibt Kredite (z.B. Bank)\pause
					\item[Schuldner] Kredit\textit{nehmer}, empfängt den Kredit (z.B. Du)
				\end{description}
			\end{frame}
		
			\begin{frame}{Zinsen}
				\begin{definition}
					Zinsen sind das Entgeld für zeitweise überlassenes Kapital (z.B. ein Kredit) vom Schuldner zum Gläubiger.\citewiki{Zins}
				\end{definition}
				Beispiele:
				\begin{multicols}{2}
					\begin{itemize}
						\item Sparbuch:\\
						\begin{tikzpicture}[->,>=stealth']
							\node (i) at (0,0) {Ich};
							\node (b) at (3,0) {Bank};
							
							\draw (i) to [bend left=10] node[above,font=\tiny] {Sparbetrag} (b);
							\draw (b) to [bend left=10] node[below,align=center,font=\tiny] {Sparbetrag\\+ Zinsen} (i);
						\end{tikzpicture}
						\columnbreak
						\item Kredit:\\
						\begin{tikzpicture}[->,>=stealth']
							\node (b) at (0,0) {Bank};
							\node (i) at (3,0) {Ich};
							
							\draw (b) to [bend left=10] node[above,font=\tiny] {Kredit} (i);
							\draw (i) to [bend left=10] node[below,align=center,font=\tiny] {Kredit\\+ Zinsen} (b);
						\end{tikzpicture}
					\end{itemize}
				\end{multicols}
			\end{frame}
		
			\begin{frame}{Zinsen}
				Die Höhe von Zinsen:
				\begin{itemize}
					\item Hängt ab von
					\begin{itemize}
						\item Bonität des Schuldners
						\item Leitzins
						\item Allgemeiner wirtschaftlicher Lage
					\end{itemize}
					\item Daumenregel: Hohe Zinsen \textrightarrow\ hohes Risiko!
					\item Zinsen = Schmerzensgeld / Belohnung für eingegangenes Risiko
				\end{itemize}
			\end{frame}
		
			\begin{frame}{Der Zinseszins-Effekt}
				\begin{definition}
					Zinsen, die auf Kapital mit zuvor hinzugefügten Zinsen gezahlt werden.\citewiki{Zinseszins}
				\end{definition}
				
				Oder: Zinsen auf Zinsen \textrightarrow\ exponentielles Wachstum\\\hfill
				
				Beispiel Sparbuch mit 10\% Zinsen:
				\begin{itemize}
					\item 2021: 100,00€ \textrightarrow\ 10,00€ Zinsen \textrightarrow\ 110,00€ am Jahresende
					\item 2022: 110,00€ \textrightarrow\ 11,00€ Zinsen \textrightarrow\ 121,00€
					\item 2022: 121,00€ \textrightarrow\ 12,10€ Zinsen \textrightarrow\ 133,10€
				\end{itemize}
			\end{frame}
	
	\section{Finanzprodukte}
	
		\begin{frame}
			\tableofcontents[currentsection,hideallsubsections]
		\end{frame}
	
		\begin{frame}{Begrifflichkeiten}
			\begin{description}[labelwidth=0cm]
				\item[Wertpapier] Urkunde, die Rechte gegenüber einem Schuldner verbrieft\citewiki{Wertpapier}
				\item[Volatilität] Größe der Schwankungen im Kursverlauf eines Wertpapiers\citewiki[Volatilität]{Volatilit\%C3\%A4t}
			\end{description}
		\end{frame}
	
		\subsection{Aktien}
		
			\begin{frame}
				\vspace{0.5cm}
				\begin{center}
					Was ist eine Aktie?
				\end{center}
			\end{frame}
			
			\begin{frame}
				\begin{itemize}
					\item Erste Aktie 1288\citewiki{Aktie}
					\item Verbrieft Anteil an Unternehmen
					\begin{itemize}
						\item Mitbesitzer des Unternehmens
						\item Aktionär
					\end{itemize}
					\item Rechte
					\begin{itemize}
						\item Stimmrecht
						\item Dividendenanspruch
					\end{itemize}
					\item Jährliche Hauptversammlung
				\end{itemize}
			\end{frame}
		
			\begin{frame}{Dividende}
				\begin{definition}
					Auszahlung eines Teils des Gewinns an die Aktionäre.\citewiki{Dividende}
				\end{definition}
				\hfill\\
				\begin{itemize}
					\item Optional (s. Alphabet Inc.)
					\item Höhe wird von Aktionären auf Hauptversammlung bestimmt
					\item Ausschüttungen meist allgemein als Dividende bezeichnet
				\end{itemize}
			\end{frame}
				
			\begin{frame}{Dividende (Beispiel)}
				Beispiel BASF:\citeurl{https://www.finanzen.net/bilanz_guv/basf}{finanzen.net}\\
				\hfill\\
				\begin{tabularx}{\linewidth}{X|X|X}
					Daten & 2018 & 2019\\
					\hline\hline
					Gewinn (mio. €)		& 4.707	& 2.737 \\
					Dividende (mio. €)	& 2.847	& 2.939 \\
					Dividende pro Aktie	& 3,20	& 3,30 \\
					Dividendenrendite	& 5,3\%	& 4,9\%
				\end{tabularx}\\
				\hfill\\
				\hfill\\\pause
				Dividendenrendite = $\frac{\text{Dividende}}{\text{Aktienkurs}}$\\
				{\tiny (Wie viel Dividende bekäme ich für $x$ € an Investition?)}
			\end{frame}
		
		\subsection{Anleihen}
		
			\begin{frame}
				\begin{definition}
					Verzinsliches Wertpapier mit Recht auf Rückzahlung und Zinsen.\citewiki{Anleihe}
				\end{definition}
				Name ist Programm: An\textbf{leihe} \textrightarrow\ Rückzahlung mit Zinsen
				\begin{itemize}
					\item Auch: Rentenpapier, Schuldverschreibung, Bond
					\item Feste Laufzeit
					\item Zinssatz (fest oder variabel)
					\item Handelbar
					\item Staatsanleihen/Unternehmensanleihen
				\end{itemize}
			\end{frame}
		
			\begin{frame}{Beispiel (BASF)}
				Anleihen \href{https://www.finanzen.net/anleihen/a1r1bn-basf-se-anleihe}{A1R1BN} und \href{https://www.finanzen.net/anleihen/a11qc7-basf-se-anleihe}{A11QC7}:
				\begin{center}
					\begin{tabularx}{8cm}{l|l|l|l}
						Anleihe	& Von			& Bis			& Zins		\\
						\hline
						\href{https://www.finanzen.net/anleihen/a1r1bn-basf-se-anleihe}{A1R1BN}	& 04.02.2014	& 03.02.2021	& 1,875\%	\\
						\href{https://www.finanzen.net/anleihen/a11qc7-basf-se-anleihe}{A11QC7}	& 12.02.2014	& 11.02.2043	& 3,250\%	\\
					\end{tabularx}
				\end{center}
			\end{frame}
		
		\subsection{Exkurs: Geschäftsanteil}
		
			\begin{frame}
				\begin{definition}
					Teil einer GmbH / Genossenschaft, der von einem Gesellschafter erworben werden kann.\citewiki{Geschäftsanteil}
				\end{definition}
			
				\begin{itemize}
					\item Beitrag zum Stammkapital (GmbH) oder Geschäftsguthabens (Genossenschaft)
					\item Dividende (z.B. GLS: 1-3\%)
					\item In der Regel: Kunde / Mitarbeiter sein
				\end{itemize}
			\end{frame}
		
		\subsection{Index}
		
			\begin{frame}
				\begin{definition}
					Kennzahl zur repräsentativen Dokumentation eines Teilmarktes.\citewiki{Aktienindex}
				\end{definition}
			
				\begin{itemize}
					\item Abstraktion auf Punktesystem
					\item Zeitliche Entwicklung spiegelt hypothetisches Portfolio wieder
					\item Kursindex:
					\begin{itemize}
						\item Aktienkurs bestimmt Index
						\item Dividenden werden ignoriert
						\item Beispiele: Dow Jones, FTSE 100, S\&P 500, Euro Stoxx 50
					\end{itemize}
					\item Performanceindex:
					\begin{itemize}
						\item Aktienkurs + Dividenden bestimmen Index
						\item Beispiel: DAX
					\end{itemize}
				\end{itemize}
			\end{frame}
		
			\begin{frame}{Beispiel (DAX)}
				\begin{itemize}
					\item DAX = \textbf{D}eutscher \textbf{A}ktieninde\textbf{x}\citewiki{DAX}
					\item Familie: MDAX, SDAX, TecDAX, DivDAX, ...
					\item Aufnahmekriterium \& Gewichtung: Streubesitz-Marktkapitalisierung
					\item Maximales Gewicht: 10\%
					\item Simple Formel:\pause
					\[
						\text{I}_t = K_T \cdot B \cdot \frac
						{
							\sum_{i=1}^{n} p_{it} \cdot q_{iT} \cdot ff_{iT} \cdot c_{it}
						}
						{
							\sum_{i=1}^{n} p_{i0} \cdot q_{i0}
						}\hspace{1cm}
					\]\pause
				\end{itemize}
				\vspace{-0.25cm}
				tl;dr \textrightarrow\ TOP 40 der deutschen Unternehmen
			\end{frame}
		
		\subsection{Fonds}
		
			\begin{frame}
				\begin{definition}
					Verwaltetes Sondervermögen, welches in festgelegte Finanzprodukte investiert wird.\citewiki{Investmentfonds}
				\end{definition}
				Oder: Sammlung an Aktien, Anleihen, etc. in die man investieren kann.
			\end{frame}
		
			\begin{frame}
				\begin{itemize}
					\item Kategorien: Länder, Branchen, Unternehmensgröße\pause
					\item Aktiv
					\begin{itemize}
						\item Fondmanager beobachtet und analysiert Markt
						\item Ziel: Besser sein als "`der Markt"'
						\item Ziel erreichen nur sehr wenige
						\item Hohe Kosten (TER): ca. 1,3-2,5\%\footnote{z.B. \href{https://www.comdirect.de/inf/fonds/LU0552028770}{Amundi Funds Emerging Markets Equity Focus (ISIN: LU0552028770)}}
					\end{itemize}\pause
					\item Passiv\pause
					\begin{itemize}
						\item Keine Anpassungen durch Fondmanager
						\item Meist Abbildung eines Index (s. \nameref{subsec:etfs})
						\item Niedrige Kosten (ca. 0,04\footnote{z.B. \href{https://de.extraetf.com/etf-profile/LU1781541096}{Lyxor Core Morningstar UK UCITS (ISIN: LU1781541096)}}-0,7\%\footnote{z.B. \href{https://de.extraetf.com/etf-profile/IE00BYPLS672}{L\&G Cyber Security UCITS (ISIN: IE00BYPLS672)}})
					\end{itemize}\pause
					\item Kein direkter Besitz der Aktien!
				\end{itemize}
			\end{frame}
		
			\begin{frame}
				\begin{tikzpicture}[->,>=stealth']
					\node (s) at (0,0) {\includegraphics[width=1.25cm,trim=0 0 0 -1.35cm]{images/stonks-guy}};
					
					\node (i) at (4.5,0) {\includegraphics[width=1.25cm]{images/bank}};
					\node[below=0cm of i,align=center] {\small Investment-\\gesellschaft};
					
					\node (b) at (9,0) {\includegraphics[width=1.25cm]{images/stockmarket}};
					\node[below of=b] {\small Börse};
					
					\onslide<2>
					{
						\money{(2,1)}
						\money{(2.1,0.9)}
						\money{(2.2,0.8)}
						\draw[thick] (s) to [bend left=10] (i);
					}
				
					\onslide<3-4>
					{
						\money{(4.5,1)}
						\draw[thick,gray!50!white] (s) to [bend left=10] (i);
					}
				
					\onslide<3>
					{
						\money{(6.7,0.9)}
						\money{(6.8,0.8)}
						\draw[thick] (i) to [bend left=10] (b);
					}
				
					\onslide<4->
					{
						\draw[thick,gray!50!white] (i) to [bend left=10] (b);
					}
				
					\onslide<4>
					{
						\draw[thick] (b) to [bend left=10] (i);
						\node at (6.7,-1) {\includegraphics[width=0.75cm]{images/stock}};
						\node at (6.85,-1.1) {\includegraphics[width=0.75cm]{images/stock}};
					}
				
					\onslide<5->
					{
						\money{(4,1)}
						\node at (5,1) {\includegraphics[width=0.75cm]{images/stock}};
						\node at (5.15,0.9) {\includegraphics[width=0.75cm]{images/stock}};
						\draw[thick,gray!50!white] (s) to [bend left=10] (i);
						\draw[thick,gray!50!white] (b) to [bend left=10] (i);
					}
				
					\onslide<5>
					{
						\draw[thick] (i) to [bend left=10] (s);
						\node at (2.25,-1) {\includegraphics[width=0.75cm]{images/fond-share}};
					}
				
					\onslide<6>
					{
						\draw[thick,gray!50!white] (i) to [bend left=10] (s);
						\node at (0,1) {\includegraphics[width=0.75cm]{images/fond-share}};
					}
				\end{tikzpicture}
			\end{frame}
		
		\subsection{ETFs}
		\label{subsec:etfs}
		
		% \item Filterung: ESG, SRI, low carbon, controversial weapons (cw)
			\begin{frame}
				ETFs = exchange traded fund (börsengehandelter Fonds)
				\begin{definition}
					Investmentfonds, die man an der Börse handeln kann.\citewiki[Börsengehandelter\_Fonds]{B\%C3\%B6rsengehandelter\_Fonds}
				\end{definition}
				Meist synonym für \textit{Indexfonds}\pause:
				\begin{definition}
					Investmentfonds, deren Zusammensetzung sich möglichst exakt an der eines Index orientiert.\citewiki{Indexfonds}
				\end{definition}
				Orientieren = Nachbilden = Abbilden = Replikation
			\end{frame}
		
			\begin{frame}{Indexfonds}
				\begin{itemize}
					\item Bildet Index ab
					\item Passiver Fond
					\item Auswahl Aktien wie im Index
				\end{itemize}\vspace{0.25cm}
				Beispiel DAX:\vspace{0.25cm}
				\begin{tabularx}{\linewidth}{X|p{3cm}|X|X}
					Unternehmen & DAX-Gewichtung & iShares\footnote{\href{https://de.extraetf.com/etf-profile/DE0005933931}{iShares Core DAX UCITS ETF (ISIN: DE0005933931)}} & Xtrackers\footnote{\href{https://de.extraetf.com/etf-profile/LU0274211480}{Xtrackers DAX UCITS ETF (ISIN: LU0274211480)}}\\
					\hline
					Linde PLC & 9,26\% & 10,71\% & 10,72\% \\
					SAP & 8,20\% & 8,72\% & 8,73\%\\
					Siemens & 7,60\% & 7,85\% & 7,85\%
				\end{tabularx}
			\end{frame}
		
			\begin{frame}{Indexfonds}
				\begin{center}
					\includegraphics[height=0.8\textheight,trim=0 0 0 0.5cm]{images/dax-etf-benchmark}\footnote{Stand 03.12.2021; Quelle: \href{https://charts.comdirect.de/charts/benchmark_underlying.chart?HEIGHT=128&HEIGHT=600&ID_BENCH1=20735&ID_BENCH2=12221463&ID_NOTATION=28520649&TIME_SPAN=5Y&WIDTH=800}{comdirect}}
				\end{center}
			\end{frame}
		
		\subsection{Derivate}
		
			\begin{frame}
				\begin{definition}
					Von normalen Finanzprodukten abgeleitetes Finanzprodukt mit fester Laufzeit.\citewiki{Derivat (Wirtschaft)}
				\end{definition}
				Arten:
				\begin{itemize}
					\item \textbf{Swaps}: Tausch von Basiswerten (z.B. Zinsen) zur Absicherung.
					\item \textbf{Futures}: Recht \& Pflicht auf festen Preis in der Zukunft.
					\item \textbf{Optionen}: Future aber ohne Pflicht (\textrightarrow\ höhere Gebühren).
					\item \textbf{Forwards}: Quasi Futures aber außerbörslich.
				\end{itemize}
			\end{frame}
		
		\subsection{Konservative Finanzprodukte}
		
			\begin{frame}
				\begin{itemize}
					\item \textbf{Kopfkissen}: Kein (negativ) Zins, keine Kosten, volle Inflation, keine Absicherung\pause
					\item \textbf{Girokonto}: Keine Zinsen, Gebühren, Einlagensicherung\pause
					\item \textbf{Tagesgeld}: Geringe Zinsen, keine Gebühren, Einlagensicherung, Überweisung nur von/an Referenzkonto\pause
					\item \textbf{Festgeld}: Mäßige Zinsen, keine Gebühren, Einlagensicherung, feste Laufzeit\pause
					\item \textbf{Sparbuch}: Kaum/keine Zinsen, keine Gebühren, Einlagensicherung, begrenztes abheben, lange Kündigungsfristen\pause
					\item \textbf{Lebensversicherung}: Kaum Zinsen, Gebühren, Einlagensicherung\pause
					\item \textbf{Gold/Edelmetalle}: Recht volatil, geringe Rendite
				\end{itemize}
			\end{frame}
		
		\subsection{Kryptowährung}
		
			\begin{frame}
				\begin{definition}
					Kryptowährung/Krypto/Coin sind digitale Vermögenswerte, die auch als Tauschmittel fungieren.\citewiki[Kryptowährung]{Kryptow\%C3\%A4hrung}
				\end{definition}
			
				\begin{itemize}
					\item Nicht reguliert
					\item Reines Spekulationsobjekt
					\item Ein Träumchen für Kriminelle (Steuerhinterziehung, Betrug)
					\item Steuerlich: Uff
					\begin{itemize}
						\item Privates Veräußerungsgeschäft (\textrightarrow\ Einkommenssteuer)
						\item Bis 600€ steuerfrei
						\item Bis 1 Jahr Haltedauer steuerpflichtig
					\end{itemize}
				\end{itemize}
			\end{frame}
	
	\section{Handelsplätze}
	
		\begin{frame}
			\tableofcontents[currentsection,hideallsubsections]
		\end{frame}
	
		\subsection{Börse}
		
			\begin{frame}
				\begin{definition}
					Organisierter Markt für standardisierte Handelsobjekte.\citewiki[Börse]{B\%Crse}
				\end{definition}
				\begin{itemize}
					\item Typische Handelsobjekte: Aktien, Anleihen, Derivate, ...
					\item Führt Angebot und Nachfrage zusammen \textrightarrow\ Entstehung Börsenkurse
					\begin{itemize}
						\item Kurse von Börse zu Börse leicht unterschiedlich
					\end{itemize}
					\item Überwacht durch z.B. BaFin
				\end{itemize}
			\end{frame}
		
			\begin{frame}
				Verschiedene Arten:
				\begin{itemize}
					\item Warenbörse
					\item Terminbörse (\textrightarrow\ Derivate)
					\item Wertpapierbörse (\textrightarrow\ Aktien, Anleihen)
					\item Energiebörse
					\item Emissionsrechtehandelssystem
				\end{itemize}
			\end{frame}
		
			\begin{frame}
				\centering
				\hspace{-0.5cm}
				\begin{tikzpicture}[->,>=stealth']
					\node (s) at (0,0) {\includegraphics[width=0.75cm,trim=0 0 0 -1.35cm]{images/stonks-guy}};
					
					\node (i) at (2.5,-2) {\includegraphics[width=0.75cm]{images/bank}};
					\node[below=-0.1cm of i, font=\scriptsize, align=center] {Bank\\(Depot)};
					
					\node (b) at (5,-4) {\includegraphics[width=0.75cm]{images/stockmarket}};
					\node[below=-0.1cm of b, font=\scriptsize] {Börse};
					
					\node (i2) at (7.5,-2) {\includegraphics[width=0.75cm]{images/bank}};
					\node[below=-0.1cm of i2, font=\scriptsize, align=center] {Bank\\(Depot)};
					
					\node (s2) at (10,0) {\reflectbox{\includegraphics[width=0.75cm,trim=0 0 0 -1.35cm]{images/stonks-guy}}};
					
					\onslide<1> {
						\draw (s) to [bend left=10] node[above right, align=center, font=\scriptsize] {Kauf-\\order} (i);
%						\smallmoney{(1.3, -0.5)}
%						\smallmoney{(1.375, -0.55)}
%						\smallmoney{(1.45, -0.6)}
%						\smallmoney{(1.525, -0.65)}
%						\smallmoney{(1.6, -0.7)}
					}
				
					\onslide<2-> {
						\draw[gray!50!white] (s) to [bend left=10] (i);
					}
					\onslide<2> {
%						\smallmoney{(2.5, -1.35)}
						
						\draw (i) to [bend left=10] node[above right, align=center, font=\scriptsize] {Kauf-\\order} (b);
%						\smallmoney{(3.875, -2.55)}
%						\smallmoney{(3.95, -2.6)}
%						\smallmoney{(4.025, -2.65)}
%						\smallmoney{(4.1, -2.7)}
					}
				
					\onslide<3-> {
						\draw[gray!50!white] (s) to [bend left=10] (i);
						\draw[gray!50!white] (i) to [bend left=10] (b);
						\node[below left=0.75cm and 0cm of b, font=\scriptsize, anchor=east] {Kauforder};
					}
					\onslide<3> {
						\draw (s2) to [bend right=10] node[above left, align=center, font=\scriptsize] {Verkaufs-\\order} (i2);
					}
				
					\onslide<4-> {
						\draw[gray!50!white] (s) to [bend left=10] (i);
						\draw[gray!50!white] (i) to [bend left=10] (b);
						\draw[gray!50!white] (s2) to [bend right=10] (i2);
					}
					\onslide<4> {
						\draw (i2) to [bend right=10] node[above left, align=center, font=\scriptsize] {Verkaufs-\\order} (b);
					}
				
					\onslide<5-> {
						\draw[gray!50!white] (s) to [bend left=10] (i);
						\draw[gray!50!white] (i) to [bend left=10] (b);
						\draw[gray!50!white] (s2) to [bend right=10] (i2);
						\draw[gray!50!white] (i2) to [bend right=10] (b);
						\node[below right=0.75cm and 0cm of b, font=\scriptsize, anchor=west] {Verkaufsorder};
					}
					\onslide<1-5> {
						\node[below right=0.075cm and -2.675cm of b, font=\tiny] {Orderbuch:};
						\node[below right=0.48cm and -2.55cm of b, minimum width=4.5cm, minimum height=0.5cm, draw, black] {};
					}
				
					\onslide<6-> {
						\node[below right=0.075cm and -2.675cm of b, font=\tiny] {Orderbuch:};
						\node[below right=0.48cm and -2.55cm of b, minimum width=4.5cm, minimum height=0.5cm, draw, green!75!black] {};
						\coordinate[below=0.75cm of b] (bb);
						\draw[<->] (bb)++(-0.5,0) -- +(1,0);
					}
					\onslide<7> {
						\smallmoney{(5, -1.45)}
						\draw (i) to [bend right=-7] (i2);
						\node at (5, -2.6) {\includegraphics[width=0.5cm]{images/stock}};
						\draw (i2) to [bend right=-7] (i);
					}
				\end{tikzpicture}
			\end{frame}
		
		\subsection{Crowd-Investing}
		
			\begin{frame}
				\begin{definition}
					Viele Anleger investieren über das Internet kleine Beträge in Unternehmen (meist  Start-ups).\citewiki{Crowdinvesting}
				\end{definition}
				Oder: Wie Crowdfunding nur mit Rendite.\\
				\hfill\\
				Aber:
				\begin{itemize}
					\item Hohe Rendite \textrightarrow\ hohes Risiko!
					\item Kaum gesetzliche Grundlagen \textrightarrow\ "`grauer Kapitalmarkt"'
				\end{itemize}
			\end{frame}
	
	\section{Privater Handel}
	
		\begin{frame}
			\tableofcontents[currentsection,hideallsubsections]
		\end{frame}
	
		\subsection{Depot, Bank, Broker}
		
			\begin{frame}{Bank}
				\begin{itemize}
					\item Einlagen-, Kredit-, Wertpapiergeschäft
					\item Wertpapierverwahrung
					\item Depotgeschäft
				\end{itemize}
			\end{frame}
		
			\begin{frame}{Depot}
				\begin{itemize}
					\item Konto für Wertpapiere
					\item Ggf. Depotführungsgebühren
					\item Angeboten von: Banken, Fondsgesellschaften, Broker
				\end{itemize}
			\end{frame}
		
			\begin{frame}{Broker}
				\begin{itemize}
					\item Engl. für Börsenmakler
					\item Finanzdienstleister (meist Banken)
					\item Fokus reiner Broker: spekulative Privatanleger
				\end{itemize}
			\end{frame}
		
		\subsection{Best practices + Tipps}
		
			\begin{frame}
				\begin{description}[labelwidth=0cm]
					\item[Diversifikation] Streuung von Anlagen zur Risikominimierung
					\item[Strategie] Einen Plan konsequent verfolgen (z.B. Core-Satellite-Strategie)
					\item[Geduld] Mehrere Jahre, alles andere wäre Spekulation
					\item[Gebühren] Depot- \& Ordergebühren, TER, Börsenplatzentgelte, Ausgabeaufschlag
					\item[Nachdenken] Clickbaits, Werbung, "`Beratung"' und "`Gruppen"'\footnote{PORSCHE CAYMAN S JUNGS! JAWOLL, JAAA! GEIL MAN!} warten an jeder Ecke
				\end{description}
			\end{frame}
	
	\section{Steuern}
	
		\begin{frame}
			\tableofcontents[currentsection,hideallsubsections]
		\end{frame}
	
		\begin{frame}{Steuern}
			\begin{center}
				\vspace{-0.5cm}
				\includegraphics[height=0.85\textheight]{images/taxes-peter-parker}
			\end{center}
		\end{frame}
	
		\begin{frame}{Steuern}
			\begin{description}[labelwidth=0cm, align=right]
				\item[Kapitalertragssteuer] 25\% auf Kapitalerträge (+Soli +Kirchensteuer)
				\item[Sparerpauschbetrag] = Freibetrag: 801€ pro Person pro Jahr
				\item[Vorabpauschale] Wird ggf. bei Fonds (z.B. ETFs) fällig
			\end{description}
		\end{frame}
	
		\begin{frame}{Steuer: HowTo}
			\begin{itemize}
				\item Abgeltungssteuer = Kapitalertragssteuer \textrightarrow\ Quellensteuer
				\item Quelle = Bank\pause
				\item Bank hält Steuer direkt ein
				\begin{itemize}
					\item Freistellungsauftrag stellen
					\item Verluste werden "`gut geschrieben"'
				\end{itemize}\pause
				\item Steuererklärung ggf. trotzdem sinnvoll
				\begin{itemize}
					\item Kein/falscher Freistellungsauftrag
					\item Sonstige Kapitalerträge
					\item Kapitalerträge im Ausland
				\end{itemize}
			\end{itemize}
		\end{frame}
	
	{
		\setbeamertemplate{background canvas}{\includegraphics[width=\paperwidth]{images/questions.png}}
		\begin{frame}[plain]
			\begin{center}
				\vspace{1.5cm}
				Fragen?
			\end{center}
		\end{frame}
		\setbeamertemplate{background canvas}{ }
	}
\end{document}













